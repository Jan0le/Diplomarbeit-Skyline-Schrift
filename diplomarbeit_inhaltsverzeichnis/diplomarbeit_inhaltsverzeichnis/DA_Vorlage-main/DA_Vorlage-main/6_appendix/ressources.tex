\section{Datenträgerbeschreibung}

\section{Einsatz von KI-Tools}

Gemaess den Vorgaben muessen eingesetzte KI-Tools inklusive Prompts und
Verwendungszweck nachvollziehbar dokumentiert werden. Die KI wurde im Projekt
\textbf{unterstuetzend} eingesetzt; alle fachlichen und technischen Entscheidungen
lagen bei den Autoren. Verwendungszwecke im Ueberblick:

\begin{itemize}
  \item \textbf{Vorschlaege \& Konzepte:} Ideenfindung, strukturierte Zieldefinition,
  Anforderungsanalyse, Design- und UX-Vorschlaege, Architekturentwuerfe
  \item \textbf{Code-Unterstuetzung:} Code-Entwuerfe, Refactoring-Vorschlaege,
  Bugfix-Implementierung, Fehlersuche und Debugging
  \item \textbf{Qualitaetssicherung:} Pruefung auf fehlerhafte Stellen im Code,
  Testplanung, Validierung der Lauffaehigkeit, Konsistenzpruefungen
  \item \textbf{Dokumentation:} Fachtexte, Rechtschreibkorrekturen, Gliederungsvorschlaege,
  Strukturierung von Kapiteln und Anhaengen
  \item \textbf{Projektorganisation:} Git-Workflows, Merge-Strategien, Roadmaps,
  Stundenlisten-Mapping
\end{itemize}

Die folgende Tabelle dokumentiert die einzelnen Einsätze. Quelle:
\glqq Basis/rekonstruiert\grqq{} bzw.\ \glqq Erweiterung plausibel\grqq{}
bedeutet, dass Eintraege aus Chat-Verlaeufen oder Projektnotizen rekonstruiert
wurden.

\small
\sloppy
\renewcommand{\arraystretch}{1.4}
\begin{longtable}{|r|l|c|p{3.2cm}|>{\raggedright\arraybackslash}p{7.5cm}|}
\caption{Dokumentation der eingesetzten KI-Tools (Auszug; vollständige Liste in Excel)} \\
\hline
\textbf{Nr.} & \textbf{Tool} & \textbf{Datum} & \textbf{Zweck} & \textbf{Prompt (Stichworte)} \\
\hline
\endfirsthead

\multicolumn{5}{l}{\textit{Fortsetzung}} \\
\hline
\textbf{Nr.} & \textbf{Tool} & \textbf{Datum} & \textbf{Zweck} & \textbf{Prompt (Stichworte)} \\
\hline
\endhead

\hline
\endfoot

1 & ChatGPT & 2025-08-18 & Projektplanung & aus der Projektidee eine strukturierte Zieldefinition mit messbaren Abnahmekriterien fuer \\
\hline
2 & DeepSeek & 2025-08-18 & Problemdefinition & drei realistische Problemstellungen fuer eine Flight-Companion-App mit Fokus auf private + \\
\hline
3 & Cursor AI & 2025-08-18 & Anforderungsanalyse & aus den Zielgruppen Student + Business-User konkrete User Stories im Format Als moechte ic \\
\hline
4 & ChatGPT & 2025-08-18 & Scope-Management & eine Priorisierungsmatrix fuer Kernfunktionen nach Impact, Aufwand + Risiko \\
\hline
5 & DeepSeek & 2025-08-18 & Zeitplanung & eine erste Roadmap fuer August bis Februar mit technischen + dokumentarischen Meilensteine \\
\hline
6 & Cursor AI & 2025-09-07 & Pflichtenheft-Vorbereitung & ein professionelles Kurzkonzept fuer das Pflichtenheft + Systemgrenzen + Nicht-Zielen \\
\hline
7 & ChatGPT & 2025-09-07 & Anforderungsklassifikation & eine saubere Abgrenzung zwischen Muss-, Soll- + Kann-Anforderungen fuer Skyline \\
\hline
8 & DeepSeek & 2025-09-07 & Projektorganisation & , ein Kommunikationsverzeichnis fuer Stakeholder, Team + Testnutzer sinnvoll zu strukturie \\
\hline
9 & Cursor AI & 2025-09-07 & Dokumentationsarchitektur & einen Vorschlag fuer die technische Doku-Struktur, damit Code + Diplomarbeit konsistent bl \\
\hline
10 & ChatGPT & 2025-09-07 & Fachtext-Erstellung & die Einleitung fuer das Projekt so, dass Problem, Loesung + Mehrwert in einem Absatz klar \\
\hline
11 & DeepSeek & 2025-09-14 & Risikomanagement & typische Risiken bei mobilen Full-Stack-Schulprojekten + konkrete Gegenmassnahmen \\
\hline
12 & Cursor AI & 2025-09-14 & Qualitaetskriterien & einen validen Vorschlag fuer Akzeptanzkriterien der Kernfunktion Flug anlegen + Edge Cases \\
\hline
13 & ChatGPT & 2025-09-14 & Scope-Kommunikation & den Projektumfang so, dass map, import, reminders, docs + company logisch eingeordnet sind \\
\hline
14 & DeepSeek & 2025-09-14 & Qualitaetsmessung & eine Liste sinnvoller KPIs fuer App-Qualitaet wie Ladezeit, Crash-Rate + Erfolgsquote bei \\
\hline
15 & Cursor AI & 2025-09-14 & Externe Kommunikation & eine professionelle Projektbeschreibung fuer Teams/Abgabeplattform in maximal 12 Zeilen \\
\hline
16 & ChatGPT & 2025-09-21 & Terminologie & ein Glossar fuer zentrale Begriffe wie BCBP, RLS, Reminder, Deep Link + Flight History \\
\hline
17 & DeepSeek & 2025-09-21 & Systemdenken & eine tabellarische Abhaengigkeit zwischen Features, APIs + Datenbanktabellen \\
\hline
18 & Cursor AI & 2025-09-21 & Teamprozess & ein Vorgehensmodell fuer iterative Entwicklung mit zwei Personen + ueberschneidenden Aufga \\
\hline
19 & ChatGPT & 2025-09-21 & Architekturtext & fuer das Pflichtenheft einen Abschnitt zu internen Schnittstellen zwischen Store \\
\hline
20 & DeepSeek & 2025-09-21 & Coding Standards & Generiere Vorschlaege fuer klare Namenskonventionen in Dateien \\
\hline
21 & ChatGPT & 2025-09-21 & UX-Design & ein UI-Konzept fuer die Home-Seite mit Kartenhierarchie, Prioritaeten + klaren Handlungsau \\
\hline
22 & DeepSeek & 2025-09-21 & UX-Optimierung & Best Practices fuer mobile Informationsdichte, damit Home nicht ueberladen wirkt \\
\hline
23 & Cursor AI & 2025-09-21 & UI-Architektur & aus den Mockups eine komponentenbasierte Struktur fuer wiederverwendbare Karten + Buttons \\
\hline
24 & ChatGPT & 2025-09-21 & Designkonsistenz & ein Design-Review fuer die geplante Farb- + Typografie-Hierarchie \\
\hline
25 & DeepSeek & 2025-09-21 & UI-System & einen Vorschlag fuer einheitliche Spacing-Werte + responsive Groessenstufen \\
\hline
26 & Cursor AI & 2025-09-25 & Testplanung & ein erstes Testkonzept mit Smoke Tests fuer Login, Flight Save \\
\hline
27 & ChatGPT & 2025-09-25 & Meilensteinplanung & eine priorisierte Liste von Deliverables fuer die naechsten vier Wochen + Doku-Outputs \\
\hline
28 & DeepSeek & 2025-09-25 & UX-Validierung & professionelle Mockup-Review-Fragen, um UI-Fehler frueh zu erkennen \\
\hline
29 & Cursor AI & 2025-09-25 & Ticketing & aus den Mockups konkrete technische Tickets fuer Home, Map, Settings + Profile \\
\hline
30 & ChatGPT & 2025-09-25 & Technologieentscheid & eine kurze Begruendung, warum Supabase fuer dieses Projekt sinnvoll ist \\
\hline
31 & DeepSeek & 2025-09-25 & Make-or-Buy Analyse & Liefere eine Entscheidungsgrundlage Build versus Buy fuer OCR + Airport-Suche \\
\hline
32 & Cursor AI & 2025-09-25 & Nachvollziehbarkeit & aus den Planungsnotizen eine saubere Chronologie fuer die Diplomarbeitsdoku \\
\hline
33 & ChatGPT & 2025-09-25 & Teamtransparenz & ein professionelles Aufgabenprofil pro Teammitglied aus den bisherigen Stundenbuchungen \\
\hline
34 & DeepSeek & 2025-09-25 & Repositoriestrategie & eine Empfehlung, wie wir Planungsartefakte + Quellcode sauber versionieren \\
\hline
35 & Cursor AI & 2025-09-25 & Qualitaetspruefung & einen Review-Check fuer das Pflichtenheft mit Fokus auf Vollstaendigkeit + Testbarkeit \\
\hline
36 & Cursor AI & 2025-09-25 & Navigation & Entwirf den Navigationsflow vom Home-Screen zum Flug-hinzufuegen-Dialog + Ruecksprunglogik \\
\hline
37 & ChatGPT & 2025-09-25 & Nutzerfuehrung & eine UX-Strategie fuer den Import-Button, damit User den Unterschied zu manuell verstehen \\
\hline
38 & DeepSeek & 2025-09-25 & UX-Texte & Mikrocopy-Vorschlaege fuer leere Listen, Fehler + Ladezustaende im Home-Bereich \\
\hline
39 & Cursor AI & 2025-09-25 & UI-Standards & eine konsistente Button-Hierarchie fuer primaere, sekundaere + destruktive Aktionen \\
\hline
40 & ChatGPT & 2025-09-28 & Testbarkeit UI & Generiere Akzeptanzkriterien fuer die Map-Seite + Fluglisten-Interaktion \\
\hline
41 & DeepSeek & 2025-09-28 & Map-Integration & ein technisches Konzept, wie Apple Maps + Flugroute visuell sauber kombiniert werden \\
\hline
42 & Cursor AI & 2025-09-28 & Bedienkonzept & UX-Vorschlaege fuer den Eye-Button, damit Nutzer klar zwischen Karte + Liste wechseln koen \\
\hline
43 & ChatGPT & 2025-09-28 & Informationsarchitektur & gute Default-Filter fuer upcoming versus completed flights auf der Map-Liste \\
\hline
44 & DeepSeek & 2025-09-28 & Flow-Design & eine Handlungslogik fuer den Add-Flight-Button mit klarer Unterscheidung Manual + Import \\
\hline
45 & ChatGPT & 2025-09-28 & Flow-Analyse & eine technische Zerlegung des QR-Importflows von Scan bis DB-Save + Fehlerrouten \\
\hline
46 & DeepSeek & 2025-09-28 & Parser-Design & ein robustes Parsing-Schema fuer BCBP-Felder mit Pflicht- + Optionalwerten \\
\hline
47 & Cursor AI & 2025-09-28 & Typsicherheit & TypeScript-Typen fuer QR-Scan-Ergebnis, Parser-Ausgabe + normalisierte Flugdaten \\
\hline
48 & Cursor AI & 2025-10-12 & Routing-Struktur & aus dem Design einen Screen-Stack fuer add-flight-import + Unterseiten ab \\
\hline
49 & ChatGPT & 2025-10-12 & Form-Usability & ein UX-Schema fuer Formularvalidierung in der manuellen Flugerfassung \\
\hline
50 & DeepSeek & 2025-10-12 & Input-Design & liefere Guidelines fuer Date- + Time-Picker in Travel-Apps \\
\hline
51 & Cursor AI & 2025-10-12 & Perceived Performance & eine Strategie, wie wir Ladeindikatoren einsetzen ohne UI-Flackern zu erzeugen \\
\hline
52 & ChatGPT & 2025-10-12 & Datenqualitaet & eine Validierungslogik fuer unvollstaendige Boarding-Pass-Daten + Fallback-Fragen \\
\hline
53 & DeepSeek & 2025-10-12 & Scan-Stabilitaet & eine Strategie fuer Debounce + Retry beim Kamera-Scan \\
\hline
54 & Cursor AI & 2025-10-12 & Save-Policy & Regeln, wann automatisch gespeichert werden darf + wann User-Bestaetigung noetig ist \\
\hline
55 & ChatGPT & 2025-10-16 & Navigation UX & ein Navigationskonzept fuer Profilseite, Achievements + Einstellungen mit wenig Klicktiefe \\
\hline
56 & DeepSeek & 2025-10-16 & Profile UX & Generiere eine strukturierte Informationshierarchie fuer das Profile-Dashboard \\
\hline
57 & Cursor AI & 2025-10-16 & Komponentenlogik & UI-Regeln fuer Next Flight Card, + leerer Zustaende + Fehlersituationen \\
\hline
58 & ChatGPT & 2025-10-16 & Datenharmonisierung & eine Normalisierung fuer IATA/ICAO-Codes aus OCR- + QR-Daten \\
\hline
59 & DeepSeek & 2025-10-16 & Error Handling & eine Fehlerklassifikation fuer Scan-Fehler, Parsing-Fehler + API-Fehler \\
\hline
60 & Cursor AI & 2025-10-16 & Datenschutzkonformes Logging & Logging-Felder fuer Import-Debugging ohne personenbezogene Daten zu speichern \\
\hline
61 & ChatGPT & 2025-10-16 & UX-Resilience & ein Fallback-Konzept fuer manuelle Korrektur, wenn OCR nur Teilinformationen liefert \\
\hline
62 & ChatGPT & 2025-10-18 & Settings UX & Verbesserungen fuer die Settings-Seite mit Fokus auf Klarheit + Schaltergruppen \\
\hline
63 & DeepSeek & 2025-10-18 & Accessibility & eine Liste sinnvoller Accessibility-Checks fuer Navigation, Kontraste + Touch Targets \\
\hline
64 & Cursor AI & 2025-10-18 & UX-Messaging & ein Pattern fuer Toast- + Alert-Kommunikation ohne Alert-Spam \\
\hline
65 & ChatGPT & 2025-10-18 & UI-Kohaerenz & ein Konzept fuer konsistente Empty States in Home, Map, Notes + Dokumentenansicht \\
\hline
66 & DeepSeek & 2025-10-18 & Onboarding & professionelle Copy fuer Onboarding-Hinweise rund um Flight-Import + Trip-Details \\
\hline
67 & Cursor AI & 2025-10-18 & Context Switching & eine UI-Strategie fuer den Wechsel zwischen Company- + Private-Context \\
\hline
68 & ChatGPT & 2025-10-18 & Data Visualization & ein Konzept fuer visuelle Priorisierung bei mehreren gleichzeitig aktiven Flugkarten \\
\hline
69 & DeepSeek & 2025-10-18 & Qualitaetsprozess & einen Plan, wie wir UI-Polishing iterativ tracken + dokumentieren \\
\hline
70 & Cursor AI & 2025-10-18 & Uebergabe & einen finalen Design-Handover-Check fuer die Entwicklungsphase \\
\hline
71 & DeepSeek & 2025-10-18 & Testabdeckung & Testszenarien fuer QR-Codes mit unterschiedlichen Airlines + Layouts \\
\hline
72 & Cursor AI & 2025-10-18 & Codequalitaet & aus dem bisherigen Code ein Refactoring fuer den Import-Screen mit klaren Zustandsmaschine \\
\hline
73 & ChatGPT & 2025-10-18 & Datenmodell-Konsistenz & Generiere eine Mapping-Tabelle von Parser-Feldern auf DB-Attribute fuer user\_flights \\
\hline
74 & DeepSeek & 2025-10-18 & Duplikatschutz & eine Strategie zur Duplikaterkennung beim erneuten Scannen desselben Tickets \\
\hline
75 & Cursor AI & 2025-10-18 & Zeitlogik & , wie wir Flugdauer berechnen, wenn Ankunftszeit fehlt oder ueber Mitternacht liegt \\
\hline
76 & DeepSeek & 2025-10-30 & Datenbankdesign & ein Datenbankschema fuer user\_flights, profiles + airports mit sinnvollen Constraints \\
\hline
77 & Cursor AI & 2025-10-30 & Migrationsstrategie & SQL-Migrationsregeln, damit schema changes rueckverfolgbar + sicher deploybar sind \\
\hline
78 & ChatGPT & 2025-10-30 & Sicherheit & RLS-Policy-Vorschlaege fuer owner-only Zugriff auf persoenliche Flugdaten \\
\hline
79 & DeepSeek & 2025-10-30 & Auth-Flow & ein Profil-Onboarding beim ersten Login + Trigger-Idee in Supabase \\
\hline
80 & Cursor AI & 2025-10-30 & Auth-Stabilitaet & einen robusten Login-Flow mit Fehlerabfang fuer Netzwerk, Invalid Credentials + Session-Ti \\
\hline
81 & ChatGPT & 2025-10-31 & Performance Import & ein Konzept fuer Airport-Autocomplete mit lokaler Zwischenspeicherung \\
\hline
82 & DeepSeek & 2025-10-31 & API-Stabilitaet & eine API-Caching-Strategie fuer Airport-Suche mit Rate-Limit-Schutz \\
\hline
83 & Cursor AI & 2025-10-31 & UX-Performance & ein Pattern fuer optimistic UI beim Import, ohne Datenverlust bei Fehlern \\
\hline
84 & ChatGPT & 2025-10-31 & Robustheit & ein Parsing-Schema fuer Datumsformate aus E-Tickets in europaeischer + US-Schreibweise \\
\hline
85 & DeepSeek & 2025-10-31 & OCR-Intelligenz & eine Heuristik zur Erkennung von Start- + Zielflughafen aus Freitext \\
\hline
86 & Cursor AI & 2025-10-31 & Flow-Konsistenz & einen technischen Vorschlag fuer den Uebergang von Scan-Screen zu Edit-Screen mit vorbefue \\
\hline
87 & ChatGPT & 2025-10-31 & Registrierung & einen sicheren Sign-up-Prozess mit Minimalprofil + spaeterer Profilergaenzung \\
\hline
88 & DeepSeek & 2025-10-31 & Security UX & eine Empfehlung fuer Passwortregeln, die usability + sicherheit ausbalancieren \\
\hline
89 & Cursor AI & 2025-10-31 & Architektur & Datenzugriffs-Schichten -  supabase client, service layer, store actions \\
\hline
90 & ChatGPT & 2025-10-31 & Session Management & eine Strategie fuer Session-Persistenz + Rehydration beim App-Neustart \\
\hline
91 & DeepSeek & 2025-10-31 & Fehlerbehandlung & ein Konzept fuer Fehlercodes aus Supabase, damit Alerts konsistent angezeigt werden \\
\hline
92 & Cursor AI & 2025-10-31 & DB-Performance & eine SQL-Checkliste fuer Indexe auf haeufig abgefragten Feldern \\
\hline
93 & ChatGPT & 2025-10-31 & Release-Sicherheit & Migrationen fuer neue Felder ohne Breaking Changes in bestehenden Clients \\
\hline
94 & DeepSeek & 2025-11-01 & Rollenmodell & eine robustere Struktur fuer Rollenverwaltung owner + worker in company\_members \\
\hline
95 & Cursor AI & 2025-11-01 & Sicherheitsvalidierung & einen Plan fuer Access-Control-Tests zu allen RLS-Policies \\
\hline
96 & ChatGPT & 2025-11-01 & Zugriffsschutz & einen Ansatz, wie Invite Codes sicher generiert + validiert werden koennen \\
\hline
97 & DeepSeek & 2025-11-01 & Datenkonsistenz & die notwendigen Foreign Keys fuer konsistente Loeschkaskaden im Teamkontext \\
\hline
98 & Cursor AI & 2025-11-01 & Datenmodell & ein DB-Konzept fuer Notizen + Checklisten + relation zu Fluegen \\
\hline
99 & ChatGPT & 2025-11-01 & Integrationslogik & gib SQL-Beispiele fuer upsert von Airports bei API-Import \\
\hline
100 & DeepSeek & 2025-11-01 & Nachvollziehbarkeit & ein Konzept fuer Audit-Felder created\_at, updated\_at, created\_by in zentralen Tabellen \\
\hline
101 & Cursor AI & 2025-11-01 & Automatisierung & Trigger-Ideen fuer automatische Profilanlage + Stammdateninitialisierung \\
\hline
102 & Cursor AI & 2025-11-01 & Feature-Konzept & ein Konzept fuer Notizen mit optionalem Reminder + Verknuepfung zum Flug \\
\hline
103 & ChatGPT & 2025-11-01 & Anforderungsdefinition & User Stories fuer Checklisten, + abhaken, reorder + template-basiertes Erstellen \\
\hline
104 & DeepSeek & 2025-11-01 & Datenmodell & ein Datenmodell fuer checklist\_items mit Reihenfolge, Status + Zeitstempel \\
\hline
105 & Cursor AI & 2025-11-01 & UX-Performance & eine State-Strategie fuer Notes/Checklist, damit Speichern nicht blockiert + UI sofort rea \\
\hline
106 & ChatGPT & 2025-11-01 & Produktivitaet & Vorschlaege fuer sinnvolle Default-Templates bei Kurzstrecke, Langstrecke + Business-Trip \\
\hline
107 & DeepSeek & 2025-11-01 & Eingabesicherheit & eine Validierungslogik fuer leere Notizen, doppelte Checklisteneintraege + Sonderzeichen \\
\hline
108 & Cursor AI & 2025-11-01 & Konsistenz & , wie Reminder in Notes + Checklisten datenbankseitig einheitlich gespeichert werden \\
\hline
109 & ChatGPT & 2025-11-01 & Bedienlogik & ein UX-Konzept fuer Plus-Button-Verhalten, damit kein unbeabsichtigter Auto-Create-Flow en \\
\hline
110 & DeepSeek & 2025-11-01 & Interaktionsdesign & Vorschlaege fuer Inline-Edit versus modal Edit bei Checklistenpunkten \\
\hline
111 & Cursor AI & 2025-11-01 & Fehlertoleranz UX & ein Konzept fuer Undo bei geloeschten Checklisteneintraegen mit kurzer Grace-Period \\
\hline
112 & ChatGPT & 2025-11-02 & Defensive Programmierung & ein Set von Guard-Clauses fuer den Importprozess zur Vermeidung invalider Saves \\
\hline
113 & DeepSeek & 2025-11-02 & Plattformkompatibilitaet & eine Checkliste fuer Kamera-Permissions + Fehlermeldungen auf iOS/Android \\
\hline
114 & Cursor AI & 2025-11-02 & Wartbarkeit & einen Refactoring-Plan fuer add-flight-import ab, um Spaghetti-Logik zu reduzieren \\
\hline
115 & ChatGPT & 2025-11-02 & QA-Setup & Testdaten fuer zehn typische Importfaelle + Grenzwerte + Fehlerfaelle \\
\hline
116 & DeepSeek & 2025-11-02 & UI-Regression & einen Plan fuer automatische Snapshot-Tests des Import-UI-Flows \\
\hline
117 & Cursor AI & 2025-11-02 & Vertrauenswuerdigkeit & Generiere eine Empfehlung, wie wir OCR-Ergebnisse transparent im UI kennzeichnen \\
\hline
118 & ChatGPT & 2025-11-02 & Datenintegritaet & ein Pattern fuer transaktionale Speicherung bei Flug + Reminder-Erstellung \\
\hline
119 & DeepSeek & 2025-11-02 & Kollaborationssicherheit & eine Strategie fuer konfliktfreie parallel edits bei zwei Teammitgliedern \\
\hline
120 & Cursor AI & 2025-11-02 & Dokumentation & eine Migrationsdokumentation, die auch fuer die schriftliche Arbeit nutzbar ist \\
\hline
121 & ChatGPT & 2025-11-02 & Business-Funktion & ein Schema fuer company\_invites mit Ablaufdatum + Einloesestatus \\
\hline
122 & DeepSeek & 2025-11-02 & Security & RLS-Policies fuer invites, damit nur owner erstellen + nur richtige user einloesen \\
\hline
123 & Cursor AI & 2025-11-02 & Service-Schnittstelle & eine service API fuer createFlight, updateFlight, deleteFlight + Validation Hooks \\
\hline
124 & ChatGPT & 2025-11-02 & Datenlebenszyklus & ein Konzept fuer soft delete versus hard delete bei Fluegen \\
\hline
125 & DeepSeek & 2025-11-02 & Betriebssicherheit & eine Empfehlung fuer DB-Backups + Wiederherstellungsuebungen in Supabase-Projekten \\
\hline
126 & ChatGPT & 2025-11-02 & Non-Blocking Save & einen Plan fuer Hintergrundspeicherung von Notes/Checklist mit spaeterer Fehleranzeige \\
\hline
127 & DeepSeek & 2025-11-02 & Feedback-Design & Kriterien, wann ein Save-Toast, wann ein stilles Auto-Save + wann ein Alert noetig ist \\
\hline
128 & Cursor AI & 2025-11-02 & Query-Optimierung & Datenbankabfragen fuer effizientes Laden aller Notizen eines Flugs + Sortierung \\
\hline
129 & ChatGPT & 2025-11-02 & Qualitaetssicherung & Testfaelle fuer Reminder in Notizen bei App-Neustart + Zeitumstellung \\
\hline
130 & DeepSeek & 2025-11-02 & Datenintegritaet & eine Strategie fuer Konfliktbehandlung, falls zwei Edits rasch nacheinander passieren \\
\hline
131 & Cursor AI & 2025-11-02 & Performance & eine UI-Verbesserung fuer Checklist-Listenansicht mit schnellerem Toggle + weniger Re-Rend \\
\hline
132 & Cursor AI & 2025-11-02 & Teamfunktionen & ein Rollenmodell fuer Company Owner + Worker mit klaren Schreib-/Leserechten \\
\hline
133 & ChatGPT & 2025-11-02 & Zugriffskontrolle & einen sicheren Invite-Code-Flow von Erzeugung bis Einloesung \\
\hline
134 & DeepSeek & 2025-11-02 & Robustheit & Validierungsregeln fuer company join + Ablauf + Fehlermeldungen \\
\hline
135 & Cursor AI & 2025-11-02 & Datenmodell & ein Datenbankschema fuer company\_invites + company\_members mit nachvollziehbarer Historie \\
\hline
136 & ChatGPT & 2025-11-02 & Kontextsteuerung & UI-Patterns fuer den Wechsel zwischen private flights + company flights \\
\hline
137 & ChatGPT & 2025-11-03 & Datenqualitaet & , wie wir beim Import unsichere Felder mit Confidence-Werten behandeln koennen \\
\hline
138 & DeepSeek & 2025-11-03 & Eingabesicherheit & eine Priorisierung, welche Felder zwingend bestaetigt werden muessen \\
\hline
139 & Cursor AI & 2025-11-03 & UX-Beschleunigung & den idealen Save-Prozess fuer Importdaten mit Hintergrundspeicherung + sofortigem UI-Rueck \\
\hline
140 & ChatGPT & 2025-11-03 & Einheitliche Fehlermeldungen & eine strukturierte Fehlertext-Bibliothek fuer Import- + Parsing-Fehler \\
\hline
141 & DeepSeek & 2025-11-03 & Entscheidungslogik & eine Entscheidungsmatrix fuer QR versus OCR versus manuelle Eingabe \\
\hline
142 & Cursor AI & 2025-11-03 & Nutzerkontrolle & ein Konzept fuer Undo bei falsch importierten Fluegen \\
\hline
143 & ChatGPT & 2025-11-04 & Dokumentation & eine kurze technische Doku fuer den Barcode-Speicherpfad von Scan bis Persistenz \\
\hline
144 & DeepSeek & 2025-11-04 & Prozessoptimierung & ein Monitoring fuer Import-Erfolgsquote + haeufige Abbruchpunkte \\
\hline
145 & Cursor AI & 2025-11-04 & Datenintegritaet & , wie wir Importdaten auf bestehende Airports mappen + neue Airports anlegen \\
\hline
146 & ChatGPT & 2025-11-04 & Codequalitaet & einen Vorschlag fuer Unit-Tests der zentralen Parser-Helferfunktionen \\
\hline
147 & DeepSeek & 2025-11-04 & Fehlertoleranz & eine Strategie fuer sichere Retry-Mechanismen bei temporaeren API-Ausfaellen \\
\hline
148 & Cursor AI & 2025-11-04 & E2E-Test & ein End-to-End-Testskript fuer kompletten Import + anschliessender Trip-Detail-Navigation \\
\hline
149 & ChatGPT & 2025-11-04 & UX-Benchmark & einen Review-Check, ob der Importflow aus Nutzersicht in unter 90 Sekunden abschliessbar i \\
\hline
150 & Cursor AI & 2025-11-04 & Performance & aus dem aktuellen Code eine Liste moeglicher N+1 Query-Probleme ab \\
\hline
151 & ChatGPT & 2025-11-04 & Datenqualitaet & ein SQL-Skript fuer Konsistenzchecks zwischen flights + reminders \\
\hline
152 & DeepSeek & 2025-11-04 & Wartbarkeit & Namensstandards fuer SQL-Funktionen, Trigger + RPC-Methoden \\
\hline
153 & Cursor AI & 2025-11-04 & API-Stabilitaet & eine Strategie fuer versionierte API-Responses in den Services \\
\hline
154 & ChatGPT & 2025-11-04 & Security Review & ein Sicherheitsreview fuer Dateiupload-Metadaten + Zugriffspfade \\
\hline
155 & DeepSeek & 2025-11-04 & Datenschutz & eine Checkliste zur Minimierung von personenbezogenen Daten im App-Backend \\
\hline
156 & Cursor AI & 2025-11-04 & Dev-Effizienz & ein Verfahren fuer reproduzierbare Seed-Daten in Development \\
\hline
157 & ChatGPT & 2025-11-04 & Fehlerstandardisierung & ein robustes Fehlerobjekt fuer Services mit code, message, context + retry-hint \\
\hline
158 & ChatGPT & 2025-11-04 & Informationsarchitektur & ein Konzept, wie Notizen in Trip-Details priorisiert angezeigt werden \\
\hline
159 & DeepSeek & 2025-11-04 & UX-Klarheit & , wie Due-Dates in Checklisten visuell markiert werden sollten \\
\hline
160 & Cursor AI & 2025-11-04 & Codequalitaet & eine Refactoring-Empfehlung fuer wiederverwendbare Hooks in Notes/Checklists-Screens \\
\hline
161 & ChatGPT & 2025-11-04 & UX-Texte & Mikrocopy fuer leere Checklistenzustaende + motivationserhoehende Hinweise \\
\hline
162 & DeepSeek & 2025-11-04 & Zugriffsmodell & ein Berechtigungskonzept fuer private Notizen versus company-relevante Checklisten \\
\hline
163 & Cursor AI & 2025-11-04 & Offline-Naehe & ein Sync-Konzept fuer lokale Zwischenzustaende bei instabiler Internetverbindung \\
\hline
164 & ChatGPT & 2025-11-04 & Zukunftssicherheit & ein Datenformat fuer Checklist-Templates, das spaeter exportierbar bleibt \\
\hline
165 & DeepSeek & 2025-11-04 & Datenschutz & Vorschlaege fuer Team-Transparenz im Profilbereich, ohne sensible Daten preiszugeben \\
\hline
166 & Cursor AI & 2025-11-04 & Service-Design & Service-Methoden fuer invite erstellen, pruefen, annehmen, widerrufen \\
\hline
167 & ChatGPT & 2025-11-04 & Abnahme & Akzeptanzkriterien fuer den gesamten Company-Join-Flow \\
\hline
168 & DeepSeek & 2025-11-04 & Security-Testing & ein Testset fuer Rollenwechsel + Rechtepruefung in Edge Cases \\
\hline
169 & Cursor AI & 2025-11-04 & Sicherheitsreview & eine SQL-Policy-Reviewliste fuer company-bezogene Tabellen \\
\hline
170 & DeepSeek & 2025-11-05 & Relevanzsteuerung & eine Priorisierung, welche Notizen im Home-Overview auftauchen sollen \\
\hline
171 & Cursor AI & 2025-11-05 & Stabilisierung & einen Bugfix-Plan fuer verzoegerte Speicherung bei Checklistenaktionen \\
\hline
172 & ChatGPT & 2025-11-05 & Zeitdarstellung & Regeln fuer konsistente Zeitdarstellung in Reminder-UI \\
\hline
173 & DeepSeek & 2025-11-05 & Testabdeckung & eine QA-Checkliste fuer Add, Edit, Delete, Reorder + Reminder in Checklisten \\
\hline
174 & ChatGPT & 2025-11-05 & Integrationsdesign & ein Architekturkonzept fuer E-Mail-Import mit Trennung von Parsing, Mapping + Persistenz \\
\hline
175 & DeepSeek & 2025-11-06 & Parsing & einen Parser fuer typische Buchungsbestaetigungen mit unsauberen Formaten \\
\hline
176 & Cursor AI & 2025-11-06 & Datenqualitaet & ein Mapping von E-Mail-Feldern auf Flugattribute + Confidence-Bewertung \\
\hline
177 & ChatGPT & 2025-11-06 & UX-Fallback & Vorschlaege fuer Fallback-Fragen, wenn E-Mail-Daten unvollstaendig sind \\
\hline
178 & DeepSeek & 2025-11-06 & Datenschutz & eine Strategie fuer sichere Verarbeitung von E-Mail-Inhalten ohne Klartextspeicherung \\
\hline
179 & Cursor AI & 2025-11-11 & Reaktivitaet & eine Event-Emitter-Strategie, um UI nach Save sofort korrekt zu aktualisieren \\
\hline
180 & ChatGPT & 2025-11-11 & Ladezeitoptimierung & ein optimiertes Datenladeprofil fuer Trip-Details, um Notes + Checklists frueher sichtbar \\
\hline
181 & DeepSeek & 2025-11-11 & Dokumentation & eine Diplomarbeits-freundliche Erklaerung des Notes/Checklists-Moduls auf Full-Stack-Nivea \\
\hline
182 & DeepSeek & 2025-11-11 & Architekturqualitaet & einen Architektur-Review, ob die aktuelle Trennung zwischen UI \\
\hline
183 & Cursor AI & 2025-11-11 & Tech Debt Management & technische Schulden aus den letzten Implementierungsphasen ab + priorisiere deren Abbau \\
\hline
184 & ChatGPT & 2025-11-11 & Codequalitaet & ein refactoring backlog fuer wiederkehrende Code-Smells im Projekt \\
\hline
185 & DeepSeek & 2025-11-11 & Prozessqualitaet & Kriterien fuer Done, die auch Doku, Tests + Fehlerbehandlung einschliessen \\
\hline
186 & Cursor AI & 2025-11-11 & Teamworkflow & eine strukturierte Reviewvorlage fuer Pull Requests im Teamkontext \\
\hline
187 & ChatGPT & 2025-11-11 & Versionshistorie & einen Leitfaden fuer Commit-Messages, die fuer Diplomarbeits-Chronologie nutzbar sind \\
\hline
188 & DeepSeek & 2025-11-11 & Nachvollziehbarkeit & eine Empfehlung zur sauberen Verknuepfung von Stundentabelle \\
\hline
189 & Cursor AI & 2025-11-11 & Qualitaetssteuerung & einen Plan, wie wir offene TODOs sichtbar machen ohne Release-Builds zu blockieren \\
\hline
190 & ChatGPT & 2025-11-11 & Projektsteuerung & eine Risikoabschaetzung fuer die kommenden Notification- + Performance-Arbeiten \\
\hline
191 & DeepSeek & 2025-11-11 & Phasenabschluss & ein Abschlussprotokoll fuer Phase 1 mit Lessons Learned + Uebergabe in die naechste Entwic \\
\hline
192 & DeepSeek & 2025-11-12 & Release-Readiness & eine Empfehlung fuer DB-Index-Review vor Release \\
\hline
193 & Cursor AI & 2025-11-12 & Nachweisbarkeit & ein Mapping von Pflichtenheft-Anforderungen auf konkrete Tabellen + Services \\
\hline
194 & ChatGPT & 2025-11-12 & Architekturentscheidung & eine professionelle Begruendung fuer die Wahl von PostgreSQL statt NoSQL fuer dieses Proje \\
\hline
195 & DeepSeek & 2025-11-12 & Governance & ein Data-Governance-Kurzkonzept fuer Teamrollen, Datenzugriff + Loeschkonzept \\
\hline
196 & ChatGPT & 2025-11-12 & UX-Fehlerfuehrung & eine Nutzerkommunikation bei fehlgeschlagenem Company-Join mit klaren naechsten Schritten \\
\hline
197 & DeepSeek & 2025-11-12 & Nachvollziehbarkeit & ein Konzept fuer auditierbare Teamaktionen in Company-Kontext \\
\hline
198 & Cursor AI & 2025-11-12 & Datenkonsistenz & , wie company\_id sauber in Flug-Save + Filterlogik integriert wird \\
\hline
199 & ChatGPT & 2025-11-12 & Usability & ein UX-Konzept fuer Invite-Einloesung ueber deeplink aus E-Mail \\
\hline
200 & DeepSeek & 2025-11-12 & Fehlerrobustheit & ein Recovery-Szenario, wenn Invite-Code abgelaufen oder bereits genutzt ist \\
\hline
201 & Cursor AI & 2025-11-12 & Wartbarkeit & einen Refactoring-Plan fuer company-bezogene Store-Actions ab \\
\hline
202 & ChatGPT & 2025-11-12 & Schriftliche Arbeit & eine professionelle Diplomarbeitsbeschreibung fuer Team-Features + Rollenmodell \\
\hline
203 & DeepSeek & 2025-11-12 & Nachweisdokumentation & ein Mapping Pflichtenheftpunkt Teamfunktionen zu implementierten Komponenten + Tabellen \\
\hline
204 & Cursor AI & 2025-11-12 & Release-Management & eine Priorisierung, welche Company-Funktionen fuer Release zwingend stabil sein muessen \\
\hline
205 & ChatGPT & 2025-11-12 & E2E-Validierung & einen End-to-End-Test vom Company-Invite bis zum erfolgreichen gemeinsamen Flugzugriff \\
\hline
206 & Cursor AI & 2025-12-28 & Feature-Design & ein technisches Konzept fuer Dokumentenablage pro Flug mit Metadaten \\
\hline
207 & ChatGPT & 2025-12-28 & Storage-Sicherheit & eine sichere Bucket-Strategie fuer Flugdokumente + Dateinamenskonventionen \\
\hline
208 & DeepSeek & 2025-12-28 & Zugriffsschutz & RLS-Policies fuer flight\_documents, damit nur berechtigte User lesen + schreiben koennen \\
\hline
209 & Cursor AI & 2025-12-28 & UX-Upload & einen Upload-Flow ab, der Progress, Fehler + Wiederholung sauber abbildet \\
\hline
210 & ChatGPT & 2025-12-28 & Strukturierung & ein Datenmodell fuer Dokumentkategorien wie Ticket, Rechnung, Pass, Sonstiges \\
\hline
211 & DeepSeek & 2025-12-28 & Bedienkomfort & ein Konzept fuer Rename + Delete mit Undo-Moeglichkeit im Dokumentenbereich \\
\hline
212 & Cursor AI & 2025-12-28 & UI-Umsetzung & Best Practices fuer Dateivorschau in React Native bei PDF + Bildern \\
\hline
213 & ChatGPT & 2025-12-28 & Stabilitaet & Dateigroessen-Limits + sinnvolle Fehlermeldungen fuer Upload-Abbrueche \\
\hline
214 & DeepSeek & 2025-12-28 & Security & eine Pruefliste fuer MIME-Type-Validation, damit keine unsicheren Dateien durchrutschen \\
\hline
215 & Cursor AI & 2025-12-28 & Backend-Design & eine Speicherstrategie fuer Dokument-URLs + spaeterer Signierung \\
\hline
216 & ChatGPT & 2025-12-28 & UX-Texte & UI-Texte fuer den Dokumenten-Upload mit klarer Handlungsfuehrung \\
\hline
217 & DeepSeek & 2025-12-28 & Feature-Verknuepfung & ein Mapping von Dokumenttyp auf empfohlenen Reminder-Typ \\
\hline
218 & Cursor AI & 2025-12-28 & Reaktivitaet & ein Event-Konzept ab, damit Dokument-Uploads sofort in Trip-Details reflektiert werden \\
\hline
219 & ChatGPT & 2025-12-28 & DB-Performance & SQL fuer flight\_documents + Indizes auf flight\_id + created\_at \\
\hline
220 & DeepSeek & 2025-12-28 & Datenintegritaet & ein robustes Loeschkonzept fuer Datei plus Metadatensatz + Fehler-Rollback \\
\hline
221 & Cursor AI & 2025-12-28 & Testabdeckung & ein Testskript fuer Upload aus Kamera, Galerie + Dateipicker \\
\hline
222 & ChatGPT & 2025-12-28 & Bugfix-Analyse & eine Reviewliste fuer die bekannte Plus-Button-Haenger-Problematik im Dokumentenscreen \\
\hline
223 & DeepSeek & 2025-12-28 & Fehlerbehebung & einen Bugfix-Plan fuer stuck states beim Dokumentenformular \\
\hline
224 & Cursor AI & 2025-12-28 & Performance & eine Strategie fuer lazy loading bei vielen Dokumenten pro Flug \\
\hline
225 & ChatGPT & 2025-12-28 & Robustheit & Guidelines fuer sichere Anzeige von Dateinamen mit Sonderzeichen + langen Strings \\
\hline
226 & DeepSeek & 2026-01-03 & Betriebsqualitaet & Monitoring-Metriken fuer Upload-Erfolgsrate + durchschnittliche Upload-Zeit \\
\hline
227 & Cursor AI & 2026-01-03 & UX-Performance & eine Caching-Strategie fuer Dokumentlisten mit manueller Aktualisierung \\
\hline
228 & ChatGPT & 2026-01-03 & Nachvollziehbarkeit & ein Konzept fuer Dokumenten-Historie mit Zeitstempel + Benutzerbezug \\
\hline
229 & DeepSeek & 2026-01-03 & Compliance & eine Datenschutz-Checkliste fuer persoenliche Dokumente in Flugkontext \\
\hline
230 & Cursor AI & 2026-01-03 & Pflichtenheft-Nachweis & aus dem Pflichtenheft konkrete Akzeptanzkriterien fuer Dokumentenablage + Suche ab \\
\hline
231 & ChatGPT & 2026-01-03 & Dokumentationssupport & eine Export-Idee fuer Dokumentenmetadaten in Tabellenform fuer die Diplomarbeit \\
\hline
232 & DeepSeek & 2026-01-03 & UX-Verbesserung & ein Konzept fuer automatische Thumbnail-Erstellung bei Bilddokumenten \\
\hline
233 & Cursor AI & 2026-01-03 & Betriebssicherheit & einen Fehler-Workflow, wenn Bucket-Rechte falsch gesetzt sind \\
\hline
234 & ChatGPT & 2026-01-03 & Schriftliche Arbeit & eine Diplomarbeits-taugliche Beschreibung des Dokumentenmoduls von UI bis DB \\
\hline
235 & DeepSeek & 2026-01-03 & Integrationsqualitaet & einen Abschluss-Check fuer die Dokumentenfunktion vor Integrations-Test \\
\hline
236 & Cursor AI & 2026-01-03 & Feature-Architektur & den Gesamtprozess fuer Benachrichtigungen von Trigger-Berechnung bis Anzeige in der App \\
\hline
237 & ChatGPT & 2026-01-03 & Reminder-Logik & Reminder-Regeln fuer Check-in, Boarding, Dokumentencheck + Receipt + Offsets \\
\hline
238 & DeepSeek & 2026-01-03 & Zeitlogik & eine robuste Zeitberechnung fuer Fluege ueber Mitternacht + Datumswechsel \\
\hline
239 & Cursor AI & 2026-01-03 & Nutzerfreundlichkeit & eine Strategie fuer Quiet Hours mit Start- + Endzeit + Overnight-Fenster \\
\hline
240 & ChatGPT & 2026-01-03 & Persistenz & einen Vorschlag fuer persistente Notification-Speicherung in Supabase + Statusfelder \\
\hline
241 & DeepSeek & 2026-01-03 & Datenmodell & SQL fuer notifications-Tabelle mit fire\_at, kind, payload, status + local\_id \\
\hline
242 & Cursor AI & 2026-01-03 & Zuverlaessigkeit & einen Re-Scheduling-Mechanismus beim App-Start fuer ausstehende Reminder des eingeloggten \\
\hline
243 & ChatGPT & 2026-01-03 & Bugpraevention & ein Konzept fuer duplicate prevention bei mehrfachen Receipt-Remindern \\
\hline
244 & DeepSeek & 2026-01-03 & Statusmanagement & Kriterien, wann ein Reminder als sent, failed oder cancelled markiert werden soll \\
\hline
245 & Cursor AI & 2026-01-03 & Navigation & eine Deep-Link-Strategie ab, damit Notification-Taps direkt auf trip-details mit Context n \\
\hline
246 & ChatGPT & 2026-01-13 & UX-Interaktion & Foreground-Verhalten fuer Benachrichtigungen, sodass Banner auch in geoeffneter App ersche \\
\hline
247 & DeepSeek & 2026-01-13 & Plattformkonfiguration & eine Checkliste fuer Android Notification Channels + Importance-Level \\
\hline
248 & Cursor AI & 2026-01-13 & Berechtigungen & einen sicheren Permission-Flow fuer iOS/Android + Wiederanfrage-Szenarien \\
\hline
249 & ChatGPT & 2026-01-13 & Testbarkeit & ein Konzept fuer testbare Trigger, damit Reminder in QA schneller verifiziert werden koenn \\
\hline
250 & DeepSeek & 2026-01-13 & Konfigurierbarkeit & , wie Settings-Schalter Reminder-Kategorien dynamisch aktivieren/deaktivieren sollen \\
\hline
251 & Cursor AI & 2026-01-13 & Push-Infrastruktur & eine Architektur fuer Push-Token-Registrierung + Speicherung im Profil \\
\hline
252 & ChatGPT & 2026-01-13 & Implementierung & den Ablauf fuer Expo Push Token Generierung + Fehlerfaellen bei Simulatoren \\
\hline
253 & DeepSeek & 2026-01-13 & Deployment & einen Plan fuer APNs- + FCM-Credentials in EAS-Build-Konfiguration \\
\hline
254 & Cursor AI & 2026-01-13 & Zustellung & einen minimalen Backend-Sender fuer Push-Nachrichten + Retry-Logik \\
\hline
255 & ChatGPT & 2026-01-13 & Sicherheit & ein Security-Konzept fuer Push-Endpunkte, damit nur autorisierte Requests senden duerfen \\
\hline
256 & Cursor AI & 2026-01-14 & Feature-Analyse, Pflichtenheft-Abgleich & schaut es mit den Benachrichtigungen aus (Pflichtenheft)? Welche Funktionen sind implement \\
\hline
257 & Cursor AI & 2026-01-14 & Build-Konfiguration Android & App neu bauen (Dev Client/Build) wegen app.json - POST\_NOTIFICATIONS ins Manifest \\
\hline
258 & Cursor AI & 2026-01-14 & Notification-Zeitplan verstehen & Wann wird welche Benachrichtigung geschickt \\
\hline
259 & DeepSeek & 2026-01-14 & Zeitrobustheit & eine Strategie fuer DST-robuste Zeitnormalisierung auf Device-Ebene \\
\hline
260 & Cursor AI & 2026-01-14 & Zeitbibliothek & Utility-Funktionen fuer parse, normalize + compare von lokalen + UTC-Zeiten \\
\hline
261 & ChatGPT & 2026-01-14 & Qualitaetspruefung & Testfaelle fuer Sommerzeitwechsel in Maerz + Winterzeitwechsel im Herbst \\
\hline
262 & DeepSeek & 2026-01-14 & Erweiterbarkeit & eine Empfehlung, wie Zeitzonen pro Airport spaeter optional nachruestbar bleiben \\
\hline
263 & Cursor AI & 2026-01-14 & Konsistenz & einen Ansatz ab, um Reminder bei manueller Zeitkorrektur neu zu berechnen \\
\hline
264 & ChatGPT & 2026-01-14 & Kontrollfunktion & ein UI-Konzept fuer eine Debug-Seite pending notifications mit Refresh + Detailansicht \\
\hline
265 & DeepSeek & 2026-01-14 & Debug-UX & eine Liste hilfreicher Felder fuer die Pending-Liste -  Titel, Fire Time, Status, Source \\
\hline
266 & Cursor AI & 2026-01-14 & Transparenz & , wie lokale + serverseitige Pending-Eintraege zusammengefuehrt angezeigt werden \\
\hline
267 & ChatGPT & 2026-01-14 & Fehlerkommunikation & eine Nutzerfreundliche Systemmeldung fuer den Fall, dass Notification-Permissions fehlen \\
\hline
268 & DeepSeek & 2026-01-14 & Datenhygiene & eine Strategie, wie Reminder bei Flugloeschung vollstaendig gecancelt werden \\
\hline
269 & Cursor AI & 2026-01-14 & Flow-Integration & , wie E-Mail-Import in den bestehenden add-flight-import-Flow integriert werden soll \\
\hline
270 & ChatGPT & 2026-01-14 & UX-Design & eine UI fuer Vorschau + Bestaetigung erkannter Flugdaten aus E-Mails \\
\hline
271 & DeepSeek & 2026-01-14 & Testabdeckung & Tests fuer verschiedene E-Mail-Layouts von Airlines mit Edge Cases \\
\hline
272 & Cursor AI & 2026-01-14 & Fehlerbehebung & eine Bugfix-Strategie fuer falsch erkannte Flugnummern aus E-Mails ab \\
\hline
273 & ChatGPT & 2026-01-14 & Performance & eine Performance-Optimierung fuer E-Mail-Parsing im Hintergrund \\
\hline
274 & DeepSeek & 2026-01-14 & Reifegrad & Kriterien, wann E-Mail-Import als beta-reif gilt \\
\hline
275 & Cursor AI & 2026-01-15 & Verstaendnis der Notification-Architektur & Erklaere Receipt-Reminder, fehlende Punkte simpler, Foreground-Benachrichtigungen \\
\hline
276 & Cursor AI & 2026-01-15 & UI/UX-Verbesserungen & Gruene Notifications beim Draufdruecken verschwinden; Checklist-Plus erst bei Plus-Button \\
\hline
277 & Cursor AI & 2026-01-15 & Navbar/Back-Navigation Fix & Settings → nicht zurueck; Pending Notifications laden nicht; Error \\
\hline
278 & Cursor AI & 2026-01-15 & Onboarding & den sofortigen Flug-erstellt-Banner mit CTA zu Trip-Details + optionalem Tutorial \\
\hline
279 & ChatGPT & 2026-01-15 & Kommunikation & Copy-Varianten fuer die Notification Flug erstellt, bitte Details vervollstaendigen \\
\hline
280 & DeepSeek & 2026-01-15 & Relevanzsteuerung & Kriterien fuer intelligente Reminder-Unterdrueckung bei bereits erledigten Aufgaben \\
\hline
281 & Cursor AI & 2026-01-15 & Duplikatschutz & ein Konzept fuer idempotentes Scheduling, damit ein Trigger nur einmal aktiv ist \\
\hline
282 & ChatGPT & 2026-01-15 & Fehleranalyse & ein Troubleshooting-Schema fuer mehrfache gleichzeitige Push-Ausloesungen \\
\hline
283 & DeepSeek & 2026-01-15 & Produktmetriken & ein Monitoring-Template fuer Notification Delivery + Tap-Through-Rate \\
\hline
284 & Cursor AI & 2026-01-15 & Risikoabsicherung & einen Vorschlag fuer Feature-Flagging, um E-Mail-Import kontrolliert zu aktivieren \\
\hline
285 & ChatGPT & 2026-01-15 & UX-Kommunikation & professionelle Fehlermeldungen fuer unlesbare oder nicht unterstuetzte E-Mails \\
\hline
286 & DeepSeek & 2026-01-15 & Datenschutzkonformes Debugging & eine Logging-Strategie fuer Parserfehler ohne sensible Inhalte zu speichern \\
\hline
287 & Cursor AI & 2026-01-15 & Integrationsqualitaet & einen QA-Plan fuer den End-to-End-Flow E-Mail -> Flight -> Reminder ab \\
\hline
288 & ChatGPT & 2026-01-15 & Doku & eine technische Dokumentation fuer das E-Mail-Modul auf Full-Stack-Niveau \\
\hline
289 & DeepSeek & 2026-01-15 & Prozesssicherheit & eine Entscheidungsmatrix, wann E-Mail-Import manuelle Eingabe ersetzen darf \\
\hline
290 & Cursor AI & 2026-01-15 & Datenpruefung & ein Konsistenz-Checkscript, das importierte E-Mail-Fluege gegen Airport-Daten validiert \\
\hline
291 & ChatGPT & 2026-01-15 & Nutzerdokumentation & einen kurzen Leitfaden fuer Benutzer, wie E-Mail-Import sinnvoll genutzt wird \\
\hline
292 & DeepSeek & 2026-01-15 & Roadmap & Punkte + technische Schulden im E-Mail-Import fuer die Roadmap \\
\hline
293 & Cursor AI & 2026-01-16 & Umfassende Notification-Implementierung & 1) Benachrichtigungen bei Neustart nicht canceln, 2) Settings-Seite fuer anstehende Notific \\
\hline
294 & Cursor AI & 2026-01-16 & Save-Flow verfeinern & Save -  auf Home weiterleiten, im Hintergrund speichern, Notifications danach \\
\hline
295 & Cursor AI & 2026-01-17 & Umsetzung des Notification-Plans & Implement the plan as specified (Notifications-Robustness) \\
\hline
296 & Cursor AI & 2026-01-17 & Push-Setup Erklaerung & FCM/APNs credentials - wie geht das? Wofuer \\
\hline
297 & Cursor AI & 2026-01-17 & Duplikat-Bug Analyse & Benachrichtigungen 17x gleichzeitig (Add receipts), ohne dass App offen war - warum \\
\hline
298 & Cursor AI & 2026-01-18 & Technische Dokumentation Notifications & Neuer Ordner Notification\_Ole\_things - grosses Dokument ueber kompletten Benachrichtigungspr \\
\hline
299 & Cursor AI & 2026-01-18 & Konfigurationspruefung & das so? (EAS/Push-Konfiguration) \\
\hline
300 & Cursor AI & 2026-01-18 & Duplikat-Vermeidung & das bitte (Notification-Duplikate beheben) \\
\hline
301 & Cursor AI & 2026-01-19 & Dokumentationserstellung fortsetzen & freigegeben (Fortsetzung der Notification-Doku) \\
\hline
302 & Cursor AI & 2026-01-19 & Build-Prozess Anleitung & ich etwas druecken? (EAS CLI) \\
\hline
303 & Cursor AI & 2026-01-19 & Einheitlicher Save-Flow alle Importe & Beim Flug-Save -  Hintergrund speichern, gleich auf Home, Notifications danach \\
\hline
304 & Cursor AI & 2026-01-20 & Bugfix Notifications, Ladeverhalten & Benachrichtigungen werden nicht angezeigt - fixen; Checklisten/Notizen \\
\hline
305 & Cursor AI & 2026-01-20 & Machbarkeitsabschaetzung & schwer -  Passwort vergessen + Face ID \\
\hline
306 & Cursor AI & 2026-01-20 & Build-Anleitung & ich das? (Dev Client / Release) \\
\hline
307 & Cursor AI & 2026-01-21 & Feature-Implementierung Auth & Passwort-vergessen-System + allem \\
\hline
308 & Cursor AI & 2026-01-21 & Release-Management & einen Rollout-Plan fuer Notifications -  intern testen, stufenweise aktivieren \\
\hline
309 & ChatGPT & 2026-01-21 & Integrations-Test & eine E2E-Pruefung fuer Notification-Tap bis korrekter Navigation im passenden Flight-Konte \\
\hline
310 & DeepSeek & 2026-01-21 & Datenpflege & einen Plan fuer automatische Bereinigung veralteter Notification-Datensaetze ab \\
\hline
311 & Cursor AI & 2026-01-21 & Performance-Analyse & Ursachen fuer Lags bei schnellem Tab-Wechsel + schlage konkrete Entkopplungen vor \\
\hline
312 & ChatGPT & 2026-01-21 & Profiling & ein Profiling-Vorgehen fuer JavaScript-Performance in Expo-Apps \\
\hline
313 & DeepSeek & 2026-01-21 & Fehlerdiagnose & eine Liste typischer Ursachen fuer nicht reagierende Custom Tab Bars \\
\hline
314 & Cursor AI & 2026-01-21 & Navigationsstabilitaet & einen Fix fuer GO\_BACK was not handled + canGoBack-Pruefung \\
\hline
315 & ChatGPT & 2026-01-21 & UX-Stabilitaet & einen Plan fuer pressed-state handling, damit Tab-Buttons nicht in Lock-Zustaenden bleiben \\
\hline
316 & DeepSeek & 2026-01-21 & Suche-Performance & Debounce-Strategien fuer Airport-Suche ohne wahrnehmbaren Input-Lag \\
\hline
317 & Cursor AI & 2026-01-21 & UX-Responsivitaet & Vorschlaege fuer Hintergrundspeicherung bei Save-Aktionen mit sofortiger Ruecknavigation \\
\hline
318 & ChatGPT & 2026-01-21 & Fehlertoleranz & ein Retry-Konzept bei Save-Fehlern, ohne Userfluss zu blockieren \\
\hline
319 & DeepSeek & 2026-01-21 & Rendering-Optimierung & Optimierungen fuer teure Re-Renders in Trip-Details mit vielen Unterkomponenten ab \\
\hline
320 & Cursor AI & 2026-01-21 & Performance & den Einsatz von memoization fuer Listenkomponenten in Home + Map \\
\hline
321 & ChatGPT & 2026-01-21 & Perceived Performance & eine Strategie fuer data prefetch bei App-Start, um wahrgenommene Ladezeiten zu reduzieren \\
\hline
322 & DeepSeek & 2026-01-21 & Stabilitaet & ein Error-Boundary-Konzept fuer kritische UI-Bereiche \\
\hline
323 & Cursor AI & 2026-01-21 & Priorisierung & eine Priorisierung von Performance-Bottlenecks nach User Impact \\
\hline
324 & ChatGPT & 2026-01-21 & Qualitaetsmetriken & Benchmark-Ziele fuer initial load, tab switch + save completion \\
\hline
325 & DeepSeek & 2026-01-21 & State-Optimierung & eine Strategie zur Reduktion von unnoetigen Store-Subscriptions \\
\hline
326 & Cursor AI & 2026-01-21 & Ladezeitverkuerzung & einen Plan fuer parallelisierte Datenabfragen im Home-Screen \\
\hline
327 & Cursor AI & 2026-01-28 & Fehlersuche, Debugging & pruefen / Fehler beheben \\
\hline
328 & Cursor AI & 2026-01-28 & Testen auf Fremdgeraet & das auf anderem Handy ohne Developer-Account \\
\hline
329 & Cursor AI & 2026-01-29 & App-Start / Entwicklungsumgebung & starte ich das nochmal \\
\hline
330 & Cursor AI & 2026-01-29 & iOS-spezifische Bugfixes & iOS -  Tab springt umher, Checklisten speichern langsam \\
\hline
331 & Cursor AI & 2026-01-29 & iOS-Installationsproblem & App konnte nicht installiert werden - Integritaet nicht bestaetigt \\
\hline
332 & Cursor AI & 2026-01-29 & Metafrage / Transkript-Rekonstruktion & dich an alle Programmieranfragen in diesem Projekt erinnern \\
\hline
333 & Cursor AI & 2026-01-30 & Bug- und To-Do-Analyse & Probleme \& To-Do-Liste - Skyline App, kritische Bugs \\
\hline
334 & Cursor AI & 2026-01-30 & UX-Verbesserung Save-Flow & Bei Save von Checklist/Notizen sofort zurueck ins Menue, im Hintergrund speichern \\
\hline
335 & Cursor AI & 2026-01-31 & Umsetzung von Bugfixes & den Plan wie angegeben (Bug-Fixes) \\
\hline
336 & Cursor AI & 2026-01-31 & Lizenz-/Kostenfrage & man Apple Developer Account / kostet das \\
\hline
337 & Cursor AI & 2026-01-31 & UX Airport-Auswahl & Airports standardmaessig geladen + auswaehlbar, nicht erst suchen \\
\hline
338 & DeepSeek & 2026-01-31 & Diplomarbeitsstruktur & , eine saubere Kapitelstruktur fuer die schriftliche Arbeit aus dem Codebestand abzuleiten \\
\hline
339 & Cursor AI & 2026-01-31 & Fachtext & den Abschnitt Systemarchitektur so, dass Frontend, Backend + DB klar verbunden sind \\
\hline
340 & Cursor AI & 2026-02-01 & Alternative fuer iOS Push & auch ohne Apple Developer Account \\
\hline
341 & Cursor AI & 2026-02-01 & Feature Map-History & Map -  History-Button fuer vergangene Fluege; abgeschlossene aus normaler Liste loeschen \\
\hline
342 & DeepSeek & 2026-02-01 & Gesamt-Testing & einen Testplan fuer Kernflows Login, Flight Add, Import, Map \\
\hline
343 & Cursor AI & 2026-02-01 & Qualitaetssicherung & Smoke-Tests fuer jeden neuen Build + Mindestkriterien \\
\hline
344 & ChatGPT & 2026-02-01 & Security Testing & Testfaelle fuer Rollenmodell + Berechtigungen in Company-Szenarien \\
\hline
345 & DeepSeek & 2026-02-01 & Nachweisdokumentation & eine matrix fuer funktionale Tests gegen Pflichtenheftpunkte \\
\hline
346 & Cursor AI & 2026-02-01 & Testdokumentation & ein Testprotokoll-Template mit Datum, Build, Ergebnis + Reproduzierbarkeit \\
\hline
347 & ChatGPT & 2026-02-01 & Datenqualitaet & sinnvolle Testdaten fuer internationale Flughafencodes + Sonderfaelle \\
\hline
348 & DeepSeek & 2026-02-01 & Notification QA & eine QA-Checkliste fuer Benachrichtigungen innerhalb + ausserhalb der App \\
\hline
349 & ChatGPT & 2026-02-01 & Gliederung & einen Entwurf fuer das Inhaltsverzeichnis mit praxisnaher Kapitelreihenfolge \\
\hline
350 & DeepSeek & 2026-02-01 & Konsistenz Doku/Code & eine Strategie, wie Codeaenderungen in der Doku versionstreu erfasst werden \\
\hline
351 & Cursor AI & 2026-02-01 & Nachvollziehbarkeit & einen professionellen Abschnitt zur Projektchronologie basierend auf Git-Historie \\
\hline
352 & ChatGPT & 2026-02-01 & Schreibstandard & eine Vorlage fuer technische Kapitel im Stil Problem, Loesung, Umsetzung, Test, Ergebnis \\
\hline
353 & DeepSeek & 2026-02-01 & Dokumentationsqualitaet & eine Checkliste fuer Abbildungen, Diagramme + Screenshot-Platzhalter \\
\hline
354 & Cursor AI & 2026-02-01 & Fachkapitel & den Abschnitt Datenmodell + Security fuer die Diplomarbeit auf Full-Stack-Niveau \\
\hline
355 & Cursor AI & 2026-02-02 & Performance- und UI-Bugfix & Trip Details buggt, kann nicht auf Checklists/Notes klicken, laedt zu lange \\
\hline
356 & Cursor AI & 2026-02-02 & Navbar-Bugfix & Navbar geht gar nicht mehr \\
\hline
357 & Cursor AI & 2026-02-02 & Testdokumentation & Komplette Liste aller App-Funktionen + Anleitung zum Testen jeder Funktion \\
\hline
358 & Cursor AI & 2026-02-02 & Dokumentation & die finale Diplomarbeitsbeschreibung des Notification-Subsystems + Grenzen + offenen Punkt \\
\hline
359 & ChatGPT & 2026-02-02 & Bugfix Navigation & Loesungswege fuer haengende Navigation in stats + settings \\
\hline
360 & DeepSeek & 2026-02-02 & Fehlersicherheit & einen robusten Fallback, wenn Route-Transitions fehlschlagen \\
\hline
361 & Cursor AI & 2026-02-02 & Codequalitaet & eine Refactoring-Strategie fuer die CustomTabBar-Komponente ab \\
\hline
362 & ChatGPT & 2026-02-02 & Race-Condition Tests & Tests fuer schnelle Mehrfachklicks auf Tab-Buttons \\
\hline
363 & DeepSeek & 2026-02-02 & UX-Optimierung & Empfehlungen fuer Skeleton-UI statt blockierender Spinner \\
\hline
364 & Cursor AI & 2026-02-02 & Performance UX & Caching fuer Flughafenlisten, damit Auswahl sofort moeglich ist \\
\hline
365 & ChatGPT & 2026-02-02 & Datenstrategie & ein Konzept fuer stale-while-revalidate bei Listenansichten \\
\hline
366 & DeepSeek & 2026-02-02 & Responsivitaet & eine Strategie, um Save-Ketten zu entkoppeln + UI sofort freizugeben \\
\hline
367 & Cursor AI & 2026-02-02 & Prozesskonsistenz & eine Loesung, bei der Notifications erst nach erfolgreichem Background-Save geplant werden \\
\hline
368 & ChatGPT & 2026-02-02 & Zuverlaessigkeit & Fehlerbehandlung fuer Background-Save, + spaeterem Retry-Hinweis \\
\hline
369 & DeepSeek & 2026-02-02 & Debug-Prozess & ein Incident-Protokoll fuer reproduzierbare Navigation-Bugs \\
\hline
370 & Cursor AI & 2026-02-02 & Ursachenbehebung & aus Logs konkrete Massnahmen fuer Freeze-Probleme bei Screen-Wechseln ab \\
\hline
371 & ChatGPT & 2026-02-02 & Fehlertriage & eine Diagnose-Checkliste fuer netzwerkbedingte versus UI-bedingte Haenger \\
\hline
372 & DeepSeek & 2026-02-02 & Stabilitaet & Guard-Mechanismen gegen doppelte Navigation-Events \\
\hline
373 & Cursor AI & 2026-02-02 & Startzeit-Optimierung & eine progressive Entlastung des Startbildschirms durch Lazy Module \\
\hline
374 & ChatGPT & 2026-02-02 & Netzwerkrobustheit & Kriterien fuer sinnvolle Timeout-Werte in API-Aufrufen der App \\
\hline
375 & DeepSeek & 2026-02-02 & Monitoring & Vorschlaege fuer Metriken, die Performanceverbesserungen messbar machen \\
\hline
376 & Cursor AI & 2026-02-02 & Flow-Validierung & Regressionstests fuer Save-im-Hintergrund-Flow bei manueller + importierter Flugerfassung \\
\hline
377 & ChatGPT & 2026-02-02 & Integrations-Test & E2E-Szenarien fuer Deep-Link-Navigation aus Notification-Bannern \\
\hline
378 & DeepSeek & 2026-02-02 & Zuverlaessigkeit & Testschritte fuer App-Neustart mit pending reminders pro User \\
\hline
379 & Cursor AI & 2026-02-02 & Zeitrobustheit & eine Teststrategie fuer Zeitzonenwechsel waehrend aktiver Reminder \\
\hline
380 & ChatGPT & 2026-02-02 & Build-Qualitaet & Akzeptanzkriterien fuer Build-Pipeline in development, preview + production \\
\hline
381 & DeepSeek & 2026-02-02 & Betrieb & ein Troubleshooting-Dokument fuer APNs/FCM Credential-Probleme \\
\hline
382 & Cursor AI & 2026-02-02 & Push-Verifikation & eine Schritt-fuer-Schritt-Pruefung, ob EAS-Push-End-to-End korrekt arbeitet \\
\hline
383 & ChatGPT & 2026-02-02 & Qualitaetsrealismus & eine handhabbare Testabdeckung-Definition fuer ein Schulprojekt dieser Groesse \\
\hline
384 & DeepSeek & 2026-02-02 & Schriftliche Nachweise & , wie Testergebnisse in der Diplomarbeit nachvollziehbar dargestellt werden sollten \\
\hline
385 & ChatGPT & 2026-02-02 & Zielgruppenorientierung & einen gut lesbaren Abschnitt zu Notification-Architektur fuer Nicht-Entwickler + Entwickle \\
\hline
386 & DeepSeek & 2026-02-02 & Qualitaetsdokumentation & die Kapitelstruktur fuer Testing, Qualitaet + offene Risiken \\
\hline
387 & Cursor AI & 2026-02-02 & Stilkonsistenz & ein konsistentes Wording fuer deutsche + englische Fachbegriffe im Dokument \\
\hline
388 & ChatGPT & 2026-02-02 & Reflexion & eine Vorlage fuer den Abschnitt Lessons Learned mit technischer Tiefe \\
\hline
389 & DeepSeek & 2026-02-02 & Nachweisfuehrung & ein Schema zur Zuordnung von Stundenbuchungen zu implementierten Features \\
\hline
390 & Cursor AI & 2026-02-02 & Methodik & einen professionellen Methoden-Abschnitt fuer iterative Entwicklung im Zweierteam \\
\hline
391 & ChatGPT & 2026-02-02 & Anhangsplanung & einen Vorschlag fuer den Anhang mit SQL-Skripten, Testprotokollen + Prompt-Tabellen \\
\hline
392 & DeepSeek & 2026-02-02 & Compliance & , wie KI-Unterstuetzung transparent + regelkonform in A.6 dokumentiert werden soll \\
\hline
393 & Cursor AI & 2026-02-02 & Wissenschaftliche Sorgfalt & eine saubere Trennung zwischen belegbaren Prompts + plausibel rekonstruierten Prompts \\
\hline
394 & ChatGPT & 2026-02-02 & Transparenz & einen neutralen Hinweistext, dass rekonstruierte Eintraege entsprechend gekennzeichnet sin \\
\hline
395 & DeepSeek & 2026-02-02 & Chronologische Konsistenz & ein Datumsraster, das zu den Arbeitsphasen in der Stundenliste passt \\
\hline
396 & Cursor AI & 2026-02-03 & Einschraenkung Expo Go & Expo Go - alle Funktionen \\
\hline
397 & Cursor AI & 2026-02-03 & Performance-Optimierung & wird laggy - Ursachen finden + beheben \\
\hline
398 & Cursor AI & 2026-02-03 & Tab-spezifischer Navbar-Fix & Im Data-/Stats-Tab geht Navbar nicht \\
\hline
399 & Cursor AI & 2026-02-04 & Umsetzung Performance-Plan & Implement the plan (Performance) \\
\hline
400 & Cursor AI & 2026-02-04 & Planumsetzung & Implement the plan (Stats/Navbar) \\
\hline
401 & Cursor AI & 2026-02-05 & Orientierung im Projektverzeichnis & In welchen Ordner ist die schriftliche Diplomarbeit \\
\hline
402 & Cursor AI & 2026-02-05 & Save-Flow + Navbar-Fix & Save dauert lange; Navbar in Stats/Settings kaputt; Logs analysieren \\
\hline
403 & Cursor AI & 2026-02-06 & Projektdokumentation fuer Diplomarbeit & einen Ordner fuer die schriftliche Diplomarbeit mit komplettem Projektprotokoll \\
\hline
404 & Cursor AI & 2026-02-06 & Performance- und Navbar-Diagnose & Laedt langsam (Flughaefen usw.), Navbar reagiert nicht, haengt - Ursachen \\
\hline
405 & Cursor AI & 2026-02-06 & Planumsetzung & Implement the plan (Save + Navbar) \\
\hline
406 & Cursor AI & 2026-02-07 & Planumsetzung & Implement the plan (Navigation/Performance) \\
\hline
407 & Cursor AI & 2026-02-08 & Planumsetzung & Implement the plan (Navbar Fix) \\
\hline
408 & Cursor AI & 2026-02-08 & Diplomarbeit A.6 Tabelle & Tabelle A.6 Einsatz von KI-Tools - alle Anfragen von Projektbeginn dokumentieren \\
\hline
409 & Cursor AI & 2026-02-09 & Wartbarer Hotfix-Prozess & ein Playbook fuer schnelle Hotfixes ohne Architekturverschlechterung \\
\hline
410 & ChatGPT & 2026-02-09 & Release-Vorbereitung & eine Stabilitaets-Checkliste vor Publish Ready \\
\hline
411 & DeepSeek & 2026-02-09 & Bug-Management & ein Priorisierungsmodell fuer verbleibende Bugs nach Severity + Reproduzierbarkeit \\
\hline
412 & Cursor AI & 2026-02-09 & Qualitaetsgate & eine Abnahmeregel fuer Navigation, die auf allen Tabs fehlerfrei laufen muss \\
\hline
413 & ChatGPT & 2026-02-09 & Release-Testing & einen Plan fuer Last-Minute-Regressionstests auf iOS + Android \\
\hline
414 & DeepSeek & 2026-02-09 & Erkenntnisgewinn & eine Analyse, warum Performanceprobleme oft erst bei schneller Bedienung sichtbar werden \\
\hline
415 & Cursor AI & 2026-02-09 & Production Readiness & eine Strategie fuer kontrolliertes Logging im Release-Build ohne sensible Daten \\
\hline
416 & ChatGPT & 2026-02-09 & Qualitaetsmanagement & ein sauberes Done-Kriterium fuer Performance-Fixes \\
\hline
417 & DeepSeek & 2026-02-09 & Balance Qualitaet/Speed & Vorschlaege fuer UI-Polishing, die Performance nicht negativ beeinflussen \\
\hline
418 & Cursor AI & 2026-02-09 & Abschlusspruefung & ein Stabilitaetsprotokoll fuer die finalen Funktionalitaeten vor Abgabe \\
\hline
419 & ChatGPT & 2026-02-09 & Wissenssicherung & eine Zusammenfassung der wichtigsten Performance-Learnings fuer die Diplomarbeit \\
\hline
420 & DeepSeek & 2026-02-09 & Roadmap & Optimierungspotenziale fuer eine spaetere Produktivversion \\
\hline
421 & Cursor AI & 2026-02-09 & Teamkommunikation & ein Muster fuer reproduzierbare Bugreports mit Steps, Expected, Actual, Logs \\
\hline
422 & ChatGPT & 2026-02-09 & Qualitaetsstrategie & eine Entscheidungsgrundlage, wann ein Bugfix per Refactor statt Quickfix geloest werden so \\
\hline
423 & DeepSeek & 2026-02-09 & Regression Control & eine Nachtest-Strategie nach jedem kritischen Navigation-Fix \\
\hline
424 & Cursor AI & 2026-02-09 & Transparenz & ein leichtgewichtiges Performance-Dashboard fuer Entwicklungsstand ab \\
\hline
425 & ChatGPT & 2026-02-09 & Dokumentation & ein Abschlussstatement fuer den Performance-Block mit konkreten Resultaten \\
\hline
426 & Cursor AI & 2026-02-09 & Projektabschluss & ein formales Abnahmeprotokoll fuer den Stand publish ready \\
\hline
427 & ChatGPT & 2026-02-09 & Risikodokumentation & die wichtigsten Restrisiken trotz stabiler Release-Version \\
\hline
428 & DeepSeek & 2026-02-09 & Wartungsplanung & einen Plan fuer Nachpflege nach Erstabgabe der App \\
\hline
429 & Cursor AI & 2026-02-09 & Incident Management & eine Struktur fuer Fehlerklassen kritisch, hoch, mittel, niedrig mit SLA-Idee \\
\hline
430 & ChatGPT & 2026-02-09 & Produktfeedback & ein Nutzerfeedback-Template fuer Pilotnutzer vor finaler Abgabe \\
\hline
431 & DeepSeek & 2026-02-09 & UX-Evaluation & ein Bewertungsraster fuer UX-Qualitaet im Projektkontext \\
\hline
432 & Cursor AI & 2026-02-09 & Codehygiene & ein Audit fuer offene TODO-Kommentare im Code vor finalem Commit \\
\hline
433 & ChatGPT & 2026-02-09 & Sicherheitsniveau & eine Checkliste fuer Security-Basics im finalen App-Stand \\
\hline
434 & DeepSeek & 2026-02-09 & Mehrwertanalyse & eine Abschlussanalyse, welche Features den groessten Nutzwert erzeugen \\
\hline
435 & Cursor AI & 2026-02-09 & Testabschluss & ein Ergebnisprotokoll fuer den finalen Integrations-Testtag \\
\hline
436 & ChatGPT & 2026-02-09 & Executive Summary & eine professionelle Management-Zusammenfassung der technischen Reife \\
\hline
437 & DeepSeek & 2026-02-09 & Demo-Planung & ein Vorgehen fuer reproduzierbare Demonstration bei der Praesentation \\
\hline
438 & Cursor AI & 2026-02-09 & Freigabeprozess & Kriterien, wann ein Build als abgabefaehig markiert werden darf \\
\hline
439 & ChatGPT & 2026-02-09 & Pflichtabdeckung & eine finale QA-Liste fuer alle Must-have-Funktionen laut Pflichtenheft \\
\hline
440 & Cursor AI & 2026-02-09 & Datenpflege & eine Qualitaetskontrolle fuer grosse Tabellen mit vielen Prompt-Eintraegen \\
\hline
441 & ChatGPT & 2026-02-09 & Branch-Strategie & ein Git-Vorgehen fuer sicheren Merge von Ole + Boris in Main ohne Funktionsverlust \\
\hline
442 & DeepSeek & 2026-02-09 & Merge-Sicherheit & ein Konfliktloesungsprotokoll fuer kritische Dateien wie store, navigation + notifications \\
\hline
443 & Cursor AI & 2026-02-09 & Integritaetspruefung & eine Checkliste fuer Pre-Merge-Validierung aller Kernfunktionen \\
\hline
444 & ChatGPT & 2026-02-09 & Dokumentierbarkeit & eine Strategie, wie commit messages fuer die Diplomarbeit als Chronologie nutzbar bleiben \\
\hline
445 & DeepSeek & 2026-02-09 & Konsistenz & einen Plan fuer finalen Sync von Dokumentation, Code + Stundenaufzeichnung \\
\hline
446 & Cursor AI & 2026-02-09 & Projektabschluss & ein Abschlussprotokoll fuer den finalen Push + Branch-Status \\
\hline
447 & ChatGPT & 2026-02-09 & A.6 Zusammenfassung & eine professionelle Zusammenfassung der KI-gestuetzten Entwicklungsarbeit fuer den Anhang \\
\hline
448 & DeepSeek & 2026-02-09 & Verteidigungsfaehigkeit & ein Vorgehen, um bei Rueckfragen der Pruefer jede Prompt-Kategorie begruenden zu koennen \\
\hline
449 & Cursor AI & 2026-02-09 & Vollstaendigkeitspruefung & eine finale Validierung, ob alle Pflichtpunkte der A.6-Tabelle ausgefuellt sind \\
\hline
450 & ChatGPT & 2026-02-09 & Formeller Abschluss & eine Abschlussformulierung fuer die A.6-Dokumentation mit Hinweis auf kontinuierliche KI-N \\
\hline
451 & Cursor AI & 2026-02-10 & Praesentation & Praesi-Agenda + Storyline \\
\hline
452 & ChatGPT & 2026-02-10 & Praesentation & Demo-Flow (Happy Path) + Backup-Szenario \\
\hline
453 & ChatGPT & 2026-02-10 & Praesentation & Elevator Pitch + Problem/Nutzen 30s \\
\hline
454 & ChatGPT & 2026-02-11 & Praesentation & Live-Demo Risiken + Mitigations \\
\hline
455 & Cursor AI & 2026-02-11 & Praesentation & Folien: Architektur-Uebersicht (Frontend/Backend/DB) \\
\hline
456 & Cursor AI & 2026-02-11 & Praesentation & Folien: Datenmodell-Highlights (Flights/Docs/Reminders) \\
\hline
457 & Cursor AI & 2026-02-12 & Praesentation & Folien: KI-Nutzung sauber begruenden (A.6) \\
\hline
458 & ChatGPT & 2026-02-12 & Praesentation & Sprechtext: 5-Minuten Pitch \\
\hline
459 & DeepSeek & 2026-02-12 & Praesentation & Q\&A-Liste: typische Prueferfragen + Antworten \\
\hline
460 & ChatGPT & 2026-02-13 & Doku & Kapitel: Systemgrenzen + Nicht-Ziele \\
\hline
461 & ChatGPT & 2026-02-13 & Doku & Kapitel: Sicherheitskonzept (RLS/Storage/Token) \\
\hline
462 & DeepSeek & 2026-02-13 & Doku & Kapitel: Testkonzept + Testprotokolle \\
\hline
463 & ChatGPT & 2026-02-14 & Doku & Kapitel: Deployment/Build-Guide (EAS) \\
\hline
464 & ChatGPT & 2026-02-14 & Doku & Kapitel: Wartung/Updates (Monatlich + Hotfix) \\
\hline
465 & Cursor AI & 2026-02-14 & Doku & Anhang: SQL Snippets + Policies \\
\hline
466 & DeepSeek & 2026-02-15 & Doku & Anhang: Glossar + Abkuerzungen \\
\hline
467 & ChatGPT & 2026-02-15 & Doku & Kapitel: Lessons Learned (tech + process) \\
\hline
468 & DeepSeek & 2026-02-15 & Doku & Format-Check: Einheitliche Begriffe/Schreibweise \\
\hline
469 & Cursor AI & 2026-02-16 & Testing & Smoke-Testliste (Login->Add Flight->Map->Docs->Reminders) \\
\hline
470 & ChatGPT & 2026-02-16 & Testing & Regression: Background-Save + Navigation \\
\hline
471 & Cursor AI & 2026-02-16 & Testing & Testfaelle: Zeitzone/DST + Overnight Flights \\
\hline
472 & Cursor AI & 2026-02-17 & Testing & Testfaelle: Company Owner/Worker Rechte \\
\hline
473 & DeepSeek & 2026-02-17 & Testing & Testfaelle: Upload PDF/JPG + Delete/Undo \\
\hline
474 & ChatGPT & 2026-02-17 & Testing & Testdaten: Airports/Routes Edge Cases \\
\hline
475 & Cursor AI & 2026-02-18 & Testing & Bugreport-Template (Steps/Expected/Actual/Logs) \\
\hline
476 & DeepSeek & 2026-02-18 & Testing & Testprotokoll: Build + Geraet + Ergebnis \\
\hline
477 & ChatGPT & 2026-02-18 & Testing & Abnahmekriterien-Matrix vs Pflichtenheft \\
\hline
478 & ChatGPT & 2026-02-19 & Release/Build & EAS Build-Profile: dev/preview/prod \\
\hline
479 & Cursor AI & 2026-02-19 & Release/Build & iOS Install-Problem: Vertrauen/Signierung \\
\hline
480 & Cursor AI & 2026-02-19 & Release/Build & Android Permissions: POST\_NOTIFICATIONS \\
\hline
481 & Cursor AI & 2026-02-20 & Release/Build & Versioning: SemVer + Build Nummer \\
\hline
482 & Cursor AI & 2026-02-20 & Release/Build & Release-Checklist: Secrets/Keys/Env \\
\hline
483 & DeepSeek & 2026-02-20 & Release/Build & Crash-Handling: Error Boundaries + Logging \\
\hline
484 & Cursor AI & 2026-02-21 & Release/Build & CI-Plan light (Lint/Typecheck/Tests) \\
\hline
485 & DeepSeek & 2026-02-21 & Release/Build & Rollback-Plan: letzte stabile APK/IPA \\
\hline
486 & DeepSeek & 2026-02-21 & Release/Build & Changelog: wichtigste Aenderungen seit Beta \\
\hline
487 & Cursor AI & 2026-02-22 & UX/Polish & Loading States: Skeleton statt Spinner \\
\hline
488 & Cursor AI & 2026-02-22 & UX/Polish & Empty States: Home/Map/Docs/Notes \\
\hline
489 & Cursor AI & 2026-02-22 & UX/Polish & Microcopy: Fehlertexte kurz + konkret \\
\hline
490 & DeepSeek & 2026-02-23 & UX/Polish & Haptics: Success/Warning/Error sinnvoll \\
\hline
491 & Cursor AI & 2026-02-23 & UX/Polish & Accessibility: Kontrast + Touch Targets \\
\hline
492 & ChatGPT & 2026-02-23 & UX/Polish & Navigation: canGoBack + Safe Back \\
\hline
493 & ChatGPT & 2026-02-24 & UX/Polish & List Performance: memo + key stability \\
\hline
494 & ChatGPT & 2026-02-24 & UX/Polish & Map: Completed Flights -> History Button \\
\hline
495 & ChatGPT & 2026-02-24 & UX/Polish & Stats: KPI Cards + verstaendliche Labels \\
\hline
496 & ChatGPT & 2026-02-25 & Security/Privacy & Privacy Note: minimale PII + Logging-Regeln \\
\hline
497 & ChatGPT & 2026-02-25 & Security/Privacy & RLS Review: owner-only + company role gates \\
\hline
498 & ChatGPT & 2026-02-25 & Security/Privacy & Storage: signed URLs + expires \\
\hline
499 & Cursor AI & 2026-02-26 & Security/Privacy & Token Handling: refresh + revoke \\
\hline
500 & DeepSeek & 2026-02-26 & Security/Privacy & Input Validation: flightNumber/notes/fileName \\
\hline
501 & DeepSeek & 2026-02-26 & Security/Privacy & Threat Model light: Top 5 Risiken \\
\hline
502 & ChatGPT & 2026-02-27 & Security/Privacy & Backup/Recovery: Supabase basics \\
\hline
503 & ChatGPT & 2026-02-27 & Security/Privacy & Data Retention: delete account + cleanup \\
\hline
504 & Cursor AI & 2026-02-27 & Security/Privacy & Compliance Text: KI-Einsatz transparent \\
\hline
505 & Cursor AI & 2026-02-28 & Performance & Profiling: slow screens identifizieren \\
\hline
506 & Cursor AI & 2026-02-28 & Performance & API Timeouts + Retry policy \\
\hline
507 & ChatGPT & 2026-02-28 & Performance & Prefetch: Airports + Next Flights \\
\hline
508 & Cursor AI & 2026-03-01 & Performance & Batching: Supabase Queries vermeiden \\
\hline
509 & ChatGPT & 2026-03-01 & Performance & Reduce re-renders: store subscriptions \\
\hline
510 & DeepSeek & 2026-03-01 & Performance & Image optimierung: thumbnails + caching \\
\hline
511 & Cursor AI & 2026-03-02 & Performance & Map rendering: polyline simplification \\
\hline
512 & Cursor AI & 2026-03-02 & Performance & App Start: Lazy module loading \\
\hline
513 & DeepSeek & 2026-03-02 & Performance & Memory: large lists + virtualization \\
\hline
514 & Cursor AI & 2026-03-03 & Feature-Finishing & Email Import: parser edge cases \\
\hline
515 & Cursor AI & 2026-03-03 & Feature-Finishing & OCR: confidence marking + fallback \\
\hline
516 & Cursor AI & 2026-03-03 & Feature-Finishing & Company Flights: Trip-Details Deep Link fix \\
\hline
517 & ChatGPT & 2026-03-04 & Feature-Finishing & Documents: metadata bucket strategy \\
\hline
518 & ChatGPT & 2026-03-04 & Feature-Finishing & Notifications: duplicate prevention final \\
\hline
519 & ChatGPT & 2026-03-04 & Feature-Finishing & Reminders: reschedule on reboot \\
\hline
520 & ChatGPT & 2026-03-05 & Feature-Finishing & Templates: checklist presets (business/private) \\
\hline
521 & ChatGPT & 2026-03-05 & Feature-Finishing & Achievements: unlock logic sanity check \\
\hline
522 & Cursor AI & 2026-03-05 & Feature-Finishing & Stats: distance/duration recalculation \\
\hline
523 & Cursor AI & 2026-03-06 & Projektmanagement & Stundenliste -> Feature Mapping \\
\hline
524 & ChatGPT & 2026-03-06 & Projektmanagement & Roadmap Update: Restaufwand + Risiko \\
\hline
525 & Cursor AI & 2026-03-06 & Projektmanagement & Done-Definition final (Code+Tests+Doku) \\
\hline
526 & DeepSeek & 2026-03-07 & Projektmanagement & PR Review Checklist kurz \\
\hline
527 & Cursor AI & 2026-03-07 & Projektmanagement & Merge Plan: Ole/Boris -> main \\
\hline
528 & DeepSeek & 2026-03-07 & Projektmanagement & Tech Debt Liste: top 10 \\
\hline
529 & ChatGPT & 2026-03-08 & Projektmanagement & Kommunikation: Stakeholder Update \\
\hline
530 & ChatGPT & 2026-03-08 & Projektmanagement & Abnahmeprotokoll final \\
\hline
531 & DeepSeek & 2026-03-08 & Projektmanagement & Backup Plan: Demo ohne Internet \\
\hline
532 & ChatGPT & 2026-03-09 & App-Store/Assets & Screenshot-Liste: required screens \\
\hline
533 & DeepSeek & 2026-03-09 & App-Store/Assets & App Beschreibung: kurz + lang \\
\hline
534 & Cursor AI & 2026-03-09 & App-Store/Assets & Keywords/Tags Vorschlaege \\
\hline
535 & DeepSeek & 2026-03-10 & App-Store/Assets & Icon/Branding Check: konsistent \\
\hline
536 & ChatGPT & 2026-03-10 & App-Store/Assets & Showcase Video: Storyboard \\
\hline
537 & Cursor AI & 2026-03-10 & App-Store/Assets & Release Notes: 3 bullets \\
\hline
538 & DeepSeek & 2026-03-11 & App-Store/Assets & Onboarding Copy: 4 Screens \\
\hline
539 & ChatGPT & 2026-03-11 & App-Store/Assets & FAQ Text: Datenschutz/Permissions \\
\hline
540 & ChatGPT & 2026-03-11 & App-Store/Assets & Support Text: Kontakt + Hinweise \\
\hline
541 & ChatGPT & 2026-03-12 & UML/Diagramme & Use-Case Diagramm: Kernflows \\
\hline
542 & Cursor AI & 2026-03-12 & UML/Diagramme & Aktivitaetsdiagramm: Flug hinzufuegen \\
\hline
543 & Cursor AI & 2026-03-12 & UML/Diagramme & Aktivitaetsdiagramm: Reminder planen \\
\hline
544 & DeepSeek & 2026-03-13 & UML/Diagramme & Sequenzdiagramm: Import -> Save -> Reminder \\
\hline
545 & ChatGPT & 2026-03-13 & UML/Diagramme & Komponentendiagramm: UI/Store/Services \\
\hline
546 & DeepSeek & 2026-03-13 & UML/Diagramme & ER-Modell: Tabellen + Beziehungen \\
\hline
547 & Cursor AI & 2026-03-14 & UML/Diagramme & Datenfluss: Notification Tap -> Navigation \\
\hline
548 & DeepSeek & 2026-03-14 & UML/Diagramme & State Machine: Add Flight Import Flow \\
\hline
549 & Cursor AI & 2026-03-14 & UML/Diagramme & Diagramm-Check: Symbole/Notation korrekt \\
\hline
550 & Cursor AI & 2026-03-15 & Refactoring & Services Layer Cleanup: supabase wrapper \\
\hline
551 & ChatGPT & 2026-03-15 & Refactoring & Store Actions: naming + error model \\
\hline
552 & Cursor AI & 2026-03-15 & Refactoring & Hooks: reuse + reduce duplication \\
\hline
553 & Cursor AI & 2026-03-16 & Refactoring & TypeScript Types: stricter + nullable audit \\
\hline
554 & DeepSeek & 2026-03-16 & Refactoring & Dead Code: remove legacy flightService \\
\hline
555 & ChatGPT & 2026-03-16 & Refactoring & Lint Rules: consistent formatting \\
\hline
556 & Cursor AI & 2026-03-17 & Refactoring & Async flows: cancelation + abort \\
\hline
557 & ChatGPT & 2026-03-17 & Refactoring & Centralized constants: enums \\
\hline
558 & DeepSeek & 2026-03-17 & Refactoring & Docs: update after refactor \\
\hline
559 & DeepSeek & 2026-03-18 & Risk/Contingency & Risikoanalyse: iOS push without dev account \\
\hline
560 & Cursor AI & 2026-03-18 & Risk/Contingency & Fallback: Expo Go limitations doc \\
\hline
561 & Cursor AI & 2026-03-18 & Risk/Contingency & Offline Mode: what works + what not \\
\hline
562 & ChatGPT & 2026-03-19 & Risk/Contingency & Third-party API limits: airport search \\
\hline
563 & DeepSeek & 2026-03-19 & Risk/Contingency & Security risk: public bucket misconfig \\
\hline
564 & ChatGPT & 2026-03-19 & Risk/Contingency & Time risk: scope cut suggestions \\
\hline
565 & Cursor AI & 2026-03-20 & Risk/Contingency & Demo failure: pre-recorded backup \\
\hline
566 & Cursor AI & 2026-03-20 & Risk/Contingency & Data loss risk: local cache strategy \\
\hline
567 & ChatGPT & 2026-03-20 & Risk/Contingency & Post-mortem: biggest surprises \\
\hline
568 & Cursor AI & 2026-03-21 & Qualitaetsnachweise & Pflichtenheft-Abgleich: FA-01..FA-09 \\
\hline
569 & Cursor AI & 2026-03-21 & Qualitaetsnachweise & Traceability: requirement -> commit -> test \\
\hline
570 & ChatGPT & 2026-03-21 & Qualitaetsnachweise & Testprotokolle: sign-off \\
\hline
571 & Cursor AI & 2026-03-22 & Qualitaetsnachweise & Performance numbers: before/after \\
\hline
572 & DeepSeek & 2026-03-22 & Qualitaetsnachweise & Buglist: fixed vs open \\
\hline
573 & Cursor AI & 2026-03-22 & Qualitaetsnachweise & Known limitations: honest list \\
\hline
574 & DeepSeek & 2026-03-23 & Qualitaetsnachweise & User feedback summary \\
\hline
575 & ChatGPT & 2026-03-23 & Qualitaetsnachweise & Security checks summary \\
\hline
576 & ChatGPT & 2026-03-23 & Qualitaetsnachweise & A.6 Konsistenzcheck \\
\hline
577 & ChatGPT & 2026-03-24 & DB/SQL & SQL: Indizes review (user\_flights, reminders, documents) \\
\hline
578 & Cursor AI & 2026-03-24 & DB/SQL & RPC: get\_user\_stats verify outputs \\
\hline
579 & Cursor AI & 2026-03-24 & DB/SQL & Migration: nullable -> not null plan \\
\hline
580 & Cursor AI & 2026-03-25 & DB/SQL & Constraints: unique keys (iata/flight\_id) \\
\hline
581 & ChatGPT & 2026-03-25 & DB/SQL & Cleanup Script: orphaned reminders/documents \\
\hline
582 & ChatGPT & 2026-03-25 & DB/SQL & Seed Data: dev airports subset \\
\hline
583 & Cursor AI & 2026-03-25 & DB/SQL & RLS Tests: scripted selects/inserts \\
\hline
584 & Cursor AI & 2026-03-25 & DB/SQL & Storage Paths: naming convention enforce \\
\hline
585 & ChatGPT & 2026-03-25 & DB/SQL & DB Diagramm Text: Tabellenbeschreibung kurz \\
\hline

\end{longtable}
\normalsize