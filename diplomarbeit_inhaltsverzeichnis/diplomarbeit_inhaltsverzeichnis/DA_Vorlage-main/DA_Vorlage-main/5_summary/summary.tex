%!TEX root = ../DA_MainDocument.tex
\section{Zusammenfassung der Ergebnisse}
\subsection{Beantwortung der vier Forschungsfragen}
Die in Kapitel 2 formulierten vier Forschungsfragen werden durch die Evaluation
(Kapitel 8) und die Modulbewertungen beantwortet:

\begin{itemize}
  \item \textbf{Weltkarte:} Die visuelle Darstellung von Flugrouten steigert
  Übersicht und Transparenz bei Vielreisenden messbar.
  \item \textbf{Import:} Die automatische Datenübernahme bringt gegenüber
  manueller Eingabe erhebliche Zeitersparnis.
  \item \textbf{Benachrichtigungen:} Proaktive Benachrichtigungen erhöhen die
  Zuverlässigkeit und Effizienz bei der Reiseorganisation deutlich.
  \item \textbf{Datenverwaltung:} Die zentralisierte und sichere Datenverwaltung
  verbessert Transparenz und Nachvollziehbarkeit von Geschäftsreisen erheblich.
\end{itemize}

\subsection{Schlussfolgerungen aus den Modulbewertungen}
Die Evaluierung der vier Module zeigt einen messbaren Mehrwert: Die Weltkarte
bietet Orientierung und Übersicht; der automatische Import entlastet Nutzer;
Benachrichtigungen reduzieren kritische Fehlzustände; die zentrale Datenverwaltung
ermöglicht vollständige Transparenz und Rückverfolgbarkeit im Unternehmenskontext.

\section{Ausblick und zukünftige Entwicklungen}
\subsection{Erweiterungsmöglichkeiten der App}
Mögliche Erweiterungen umfassen die Integration weiterer Transportmittel (Zug, Hotel), erweiterte Analytics-Funktionen und verbesserte Team-Kollaboration.

\subsection{Verbesserungspotenziale}
Verbesserungspotenziale liegen in der Erweiterung der Testabdeckung, der Optimierung der Performance bei sehr großen Datenmengen und der Erweiterung der Import-Funktionen.
