%!TEX root = ../DA_MainDocument.tex
\chapter{Technische Umsetzung und Architektur}\label{chapter:technik}
Dieses Kapitel beschreibt die technische Architektur von Skyline und fasst die
Umsetzung der in Kapitel 4 bis 7 detailliert behandelten Module aus technischer
Sicht zusammen. Skyline wurde primär als \textbf{native iOS-App} entwickelt und
ist im \textbf{Apple App Store} verfügbar. Android wird als sekundäre
Zielplattform unterstützt, wurde im Projektzeitraum jedoch weniger umfangreich
getestet. Weitere projektbezogene Details finden sich in
Abschnitt~\ref{sec:projektumsetzung}.

\section{Systemarchitektur}
Die Architektur folgt einem Client-Server-Modell mit mobiler App und Supabase
als Backend. Die App kommuniziert über Services mit der Datenbank und dem
Storage \cite{supabasePlatform}.
\begin{figure}[h]
  \centering
  \fbox{\parbox{0.8\textwidth}{TODO: Architekturdiagramm (Client, Services, Supabase) einfügen.}}
  \caption{TODO: Systemarchitektur von Skyline}
  \label{fig:todo_architektur}
\end{figure}
\subsection{Client (React Native + Expo, iOS-primär)}
Der Client basiert auf React Native und Expo, nutzt Expo Router und eine
modulare Komponentenstruktur \cite{reactNative,expoDocs}. Der
\textbf{iOS-Build} wird über EAS Build erzeugt und im Apple App Store
bereitgestellt; ein Android-Build existiert, wurde aber im Projektzeitraum
deutlich weniger getestet.
\subsection{Backend (Supabase + Postgres)}
Supabase liefert Authentifizierung, Postgres-Datenbank, Realtime und RLS
\cite{supabasePlatform,supabaseAuth,supabaseRLS}.
\subsection{Storage (Dokumente, Bilder)}
Dokumente und Bilder werden in Storage-Buckets abgelegt und über signierte
URLs bereitgestellt \cite{supabaseStorage}.
\section{Technologien}
Die Technologieauswahl orientiert sich am Pflichtenheft und an mobiler
Cross-Platform-Entwicklung.
\subsection{Frontend-Stack}
React Native, Expo, TypeScript, Zustand für State-Management und
React Native Maps für die Karte \cite{reactNative,expoDocs,reactNativeMaps}.
Der Primär-Deployment-Kanal ist der \textbf{Apple App Store} (iOS); Android
ist als sekundäre Plattform vorgesehen.
\subsection{Backend-Stack}
Supabase, Postgres, RLS-Policies und Storage für Dateiverwaltung
\cite{supabasePlatform,supabaseRLS,supabaseStorage}.
\subsection{APIs (Aviationstack, OCR, Maps, Email-Import)}
Aviationstack liefert Airport-Daten, OCR extrahiert Text aus Dokumenten und
Maps-APIs stellen Kartendaten bereit \cite{aviationstack}.
\section{Funktionalitaet}
Die App deckt Flüge, Dokumente, Stats, Notifications und Company-Funktionen ab.
\subsection{Flugverwaltung (CRUD)}
Flüge können erstellt, bearbeitet und gelöscht werden; Zeiten und Distanzen
werden automatisch berechnet.
\subsection{Import (QR/OCR/E-Mail)}
Importfunktionen erlauben die schnelle Übernahme von Flugdaten.
\subsection{Map \& Animation}
Flugrouten werden geodätisch visualisiert und mit Animationen ergänzt.
\subsection{Notifications}
Lokale Benachrichtigungen mit Offsets und Quiet Hours unterstützen die
Reiseorganisation. Push-Benachrichtigungen laufen über APNs (iOS),
da Skyline primär für iOS entwickelt wurde; FCM (Android) ist als
künftige Erweiterung vorgesehen, aber noch nicht vollständig getestet.
\subsection{Dokumente / Notizen / Checklisten}
Reisende können Dokumente hochladen und Notizen/Checklisten pflegen.
\subsection{Statistiken \& Export}
Statistiken zeigen Distanz, Länder und Flugzeiten; CSV-Export ist möglich.
Die Berechnungen basieren auf den in der Flugverwaltung gespeicherten Daten;
Distanzen werden mittels Haversine-Formel ermittelt (Kapitel 6).
\subsection{Company-Features (Invite, Join, Dashboard)}
Unternehmensfunktionen ermöglichen Team-Management, Einladungen und gemeinsame
Übersichten. Die Details sind in der projektbezogenen Umsetzung
(Abschnitt~\ref{sec:projektumsetzung}) dokumentiert.
