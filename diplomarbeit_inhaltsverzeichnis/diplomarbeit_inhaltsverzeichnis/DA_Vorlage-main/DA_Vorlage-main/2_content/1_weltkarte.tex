%!TEX root = ../DA_MainDocument.tex
\chapter{Implementierung: Interaktive Weltkarte \& Routenvisualisierung (Boris)}\label{chapter:weltkarte}

Dieses Kapitel beschreibt die Umsetzung der Kartenvisualisierung in Skyline. Die theoretischen Grundlagen (Haversine, Great Circle, Anforderungen an mobile Karten) werden in Kapitel 3 behandelt.

\section{Implementierung der Karten-Visualisierung}
Die Implementierung kombiniert Map-Komponenten, Routenberechnung und UI-Interaktion. Ausgewaehlte Flüge werden hervorgehoben, die Route wird gezeichnet und optional animiert.
\begin{figure}[h]
  \centering
  \fbox{\parbox{0.8\textwidth}{TODO: Screenshot der Weltkarte mit Route und Marker einfuegen.}}
  \caption{TODO: Interaktive Weltkarte mit Flugroute}
  \label{fig:todo_weltkarte}
\end{figure}
\subsection{Polyline-Darstellung}
Routen werden mit Polylines gezeichnet. Die Punkte der Linie ergeben sich aus geodaetischer Interpolation zwischen Start und Ziel.
\subsection{Geodaetische Kurven}
Die Kurvenform entsteht durch das Sampling einer Great-Circle-Route in viele Zwischenpunkte, die als Polyline gerendert werden.
\subsection{Live-Marker \& Flugbewegung}
Ein Flugzeug-Marker wird entlang der Route positioniert. Bei aktiven Flügen wird er regelmaessig aktualisiert, um den Fortschritt abzubilden.
\subsection{Fortschritts-Overlay (zurueckgelegte Flugdistanz)}
Optional wird die bereits zurueckgelegte Strecke visuell markiert.
\subsection{Interaktive Flugauswahl (Map -> Details)}
Die Auswahl eines Flugs fuehrt in die Detailansicht.
\subsection{History vs. Upcoming-Flüge}
Vergangene Flüge werden von bevorstehenden getrennt, um die Karte nicht zu überladen.

\section{Bewertung der Kartenloesung}
Die Bewertung beruecksichtigt Bedienbarkeit, Genauigkeit und Performance.
\subsection{Kriterienkatalog}
Die Karte soll eine klare Übersicht über Routen liefern, ohne Informationsflut.
\subsection{Performance-Messungen}
Messungen betreffen Ladezeiten, Renderzeiten der Polylines und Reaktionszeit bei Interaktionen.
\subsection{Nutzerfeedback}
Rueckmeldungen aus Tests zeigen, ob die Karte als hilfreich und intuitiv wahrgenommen wird.
\subsection{Staerken-Schwaechen-Analyse}
Staerken liegen in der visuellen Übersicht und der Interaktion, Schwaechen entstehen bei sehr vielen Flügen.
\subsection{Ergebnis}
Die Kartenvisualisierung erfuellt die Kernanforderungen des Pflichtenhefts und stellt einen zentralen Mehrwert des Projekts dar.
