%!TEX root = ../DA_MainDocument.tex
\chapter{Interaktive Weltkarte \& Routenvisualisierung (Boris)}\label{chapter:weltkarte}
\section{Grundlagen visueller Flugrouten-Darstellung}
Die visuelle Darstellung von Flugrouten dient der Orientierung und der schnellen
Übersicht über Reisehistorie und geplante Flüge. In Skyline werden Routen
geografisch korrekt auf einer Weltkarte dargestellt und mit Interaktion
verbunden, um Details pro Flug abrufen zu können.
\section{Anforderungen an eine mobile Flugrouten-Karte}
Die Karte muss auf mobilen Geraeten performant laufen, interaktiv bedienbar sein
und darf die Nutzerfuehrung nicht überladen. Gleichzeitig sollen Routen,
Marker und Detailinformationen klar erkennbar sein.
\subsection{Performance \& Ladezeiten}
Routenberechnung, Marker-Rendering und Animationen müssen schnell geladen werden,
um Hakeleffekte zu vermeiden. Daher werden Daten reduziert geladen und
Berechnungen moeglichst client-seitig effizient ausgefuehrt.
\subsection{Skalierbarkeit bei vielen Flügen}
Bei vielen Flügen steigt die Datenmenge auf der Karte. Es braucht eine
geordnete Darstellung (z. B. Filter/History) und robuste Logik, um Ueberladung
zu vermeiden.
\subsection{Mobile Optimierung (iOS Karten)}
Die Karte muss sich an Touch-Bedienung, verschiedene Displaygroessen und das
Verhalten der nativen Karten-APIs anpassen. Skyline nutzt React Native Maps,
was eine plattformnahe Darstellung erlaubt \cite{reactNativeMaps}.
\subsection{Genauigkeit der Darstellung}
Die Route soll geodaetisch korrekt verlaufen, nicht als gerade Linie.
Ziel ist eine realistische Visualisierung, die den kuerzesten Weg auf der
Erdoberflaeche abbildet \cite{movableTypeLatLong}.
\subsection{Usability: Fokus, Zoom, Auswahl}
Wesentlich ist die einfache Auswahl eines Flugs durch Antippen sowie das
fokussierte Zoomen auf Route oder Marker. Die Interaktion soll intuitiv sein
und keine separate Navigation erfordern.
\section{Technische Grundlagen der Routenberechnung}
Die Routen werden anhand geographischer Koordinaten der Airports berechnet.
Zusätzlich wird ein Live-Fortschritt über Zeitstempel abgeleitet.
\subsection{Haversine-Formel (Distanzberechnung)}
Fuer die Distanz zwischen zwei Airports wird die Haversine-Formel verwendet,
die die Erdkruemmung beruecksichtigt und realistische Kilometerwerte liefert
\cite{movableTypeLatLong}.
\subsection{Geodaetische Routen (Great Circle)}
Die Flugroute wird als Great-Circle-Bogen modelliert. Dadurch entsteht eine
natuerliche Kurve auf der Karte statt einer linearen Verbindung
\cite{movableTypeLatLong}.
\subsection{Flugfortschritt in Echtzeit}
Wenn ein Flug gerade stattfindet, wird die Position des Flugzeugs anhand des
Verhaeltnisses zwischen departureAt und arrivalAt interpoliert.
\subsection{Zeitliche Zuordnung (departureAt/arrivalAt)}
Die Verwendung realer Zeitstempel ermoeglicht eine zeitbasierte Darstellung
von Flugfortschritt und eine realistische Live-Position.
\section{Implementierung der Karten-Visualisierung}
Die Implementierung kombiniert Map-Komponenten, Routenberechnung und UI-Interaktion.
Ausgewaehlte Flüge werden hervorgehoben, die Route wird gezeichnet und optional
animiert.
\begin{figure}[h]
  \centering
  \fbox{\parbox{0.8\textwidth}{TODO: Screenshot der Weltkarte mit Route und Marker einfuegen.}}
  \caption{TODO: Interaktive Weltkarte mit Flugroute}
  \label{fig:todo_weltkarte}
\end{figure}
\subsection{Polyline-Darstellung}
Routen werden mit Polylines gezeichnet. Die Punkte der Linie ergeben sich aus
geodaetischer Interpolation zwischen Start und Ziel.
\subsection{Geodaetische Kurven}
Die Kurvenform entsteht durch das Sampling einer Great-Circle-Route in viele
Zwischenpunkte, die als Polyline gerendert werden.
\subsection{Live-Marker \& Flugbewegung}
Ein Flugzeug-Marker wird entlang der Route positioniert. Bei aktiven Flügen
wird er regelmaessig aktualisiert, um den Fortschritt abzubilden.
\subsection{Fortschritts-Overlay (zurueckgelegte Flugdistanz)}
Optional wird die bereits zurueckgelegte Strecke visuell markiert, indem ein
Teil der Route anders eingefaerbt oder überlagert wird.
\subsection{Interaktive Flugauswahl (Map -> Details)}
Die Auswahl eines Flugs fuehrt in die Detailansicht, um Informationen wie
Zeiten, Dokumente und Notizen einzusehen.
\subsection{History vs. Upcoming-Flüge}
Vergangene Flüge werden von bevorstehenden getrennt, um die Karte nicht zu
überladen und eine klare Nutzerfuehrung zu ermoeglichen.
\section{Bewertung der Kartenloesung}
Die Bewertung beruecksichtigt Bedienbarkeit, Genauigkeit und Performance.
Zusätzlich wird Nutzerfeedback in die Analyse einbezogen.
\subsection{Kriterienkatalog (Übersicht, Transparenz, Verstaendlichkeit)}
Die Karte soll eine klare Übersicht über Routen liefern, ohne Informationsflut.
Transparenz entsteht durch eindeutige Marker und stabile Interaktionen.
\subsection{Performance-Messungen}
Messungen betreffen Ladezeiten, Renderzeiten der Polylines und die Reaktionszeit
bei Interaktionen. Ziel ist eine fluessige Bedienung.
\subsection{Nutzerfeedback}
Rueckmeldungen aus Tests zeigen, ob die Karte als hilfreich und intuitiv
wahrgenommen wird.
\subsection{Staerken-Schwaechen-Analyse}
Staerken liegen in der visuellen Übersicht und der Interaktion, Schwaechen
entstehen bei sehr vielen Flügen oder geringer Datenqualitaet.
\subsection{Ergebnis}
Die Kartenvisualisierung erfuellt die Kernanforderungen des Pflichtenhefts und
stellt einen zentralen Mehrwert des Projekts dar.
