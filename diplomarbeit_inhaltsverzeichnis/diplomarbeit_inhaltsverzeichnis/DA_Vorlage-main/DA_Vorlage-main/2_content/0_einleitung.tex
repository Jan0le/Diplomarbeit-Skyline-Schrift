%!TEX root = ../DA_MainDocument.tex
\chapter{Einleitung}

\section{Ausgangssituation und Motivation}
Reisebezogene Informationen wie Buchungsdaten, Boardingkarten und Belege werden
in der Praxis häufig in unterschiedlichen Medien und Systemen gespeichert
(E-Mail, Dateien, Kalender, Papier). Diese Verteilung erschwert die strukturierte
Organisation, verlängert Suchvorgänge und reduziert die Nachvollziehbarkeit
im Unternehmenskontext. Die Diplomarbeit adressiert diese Fragmentierung durch
eine zentrale, mobile Anwendung, die Flüge, Reisedaten, Dokumente und
Erinnerungen konsistent zusammenführt.

\section{Problemstellung}
Die zentrale Problemstellung ist die fehlende Transparenz und Nachvollziehbarkeit
von Reiseinformationen bei gleichzeitig hoher Zeitkritikalität (z.\,B. Check-in,
Boarding, Belegverwaltung). Ohne strukturierte Ablage und proaktive Hinweise
entstehen organisatorische Fehler und ein hoher manueller Aufwand. Daraus
ergeben sich Anforderungen an Importprozesse, Datenhaltung, Sicherheit,
Visualisierung und Benachrichtigung.
Die Verarbeitung personenbezogener Reisedaten muss zudem den Vorgaben der DSGVO
entsprechen \cite{gdpr}.

\section{Zielsetzung der Arbeit}
Ziel ist die Umsetzung der App \glqq Skyline\grqq{} als integrierte Lösung für
Flugverwaltung, Dokumentenablage, automatischen Import, Kartenvisualisierung,
Statistiken und Erinnerungen. Die Arbeit dokumentiert die Implementierung
strukturiert und bewertet die Wirkung in Bezug auf Zuverlässigkeit,
Effizienz, Transparenz und Nachvollziehbarkeit.

\section{Forschungsfragen}
Die Arbeit orientiert sich an folgenden vier Forschungsfragen, die den Modulen
der Implementierung zugeordnet sind:

\begin{itemize}
  \item \textbf{Weltkarte (Boris):} Wie sehr steigert eine visuelle Darstellung
  die Übersicht und Transparenz bei Vielreisenden?
  \item \textbf{Import (Boris):} Wie viel Zeitersparnis bringt die automatische
  Datenübernahme im Vergleich zur manuellen Eingabe?
  \item \textbf{Benachrichtigungen (Jan-Ole):} Wie sehr erhöhen proaktive
  Benachrichtigungen die Zuverlässigkeit und Effizienz bei der Reiseorganisation?
  \item \textbf{Datenverwaltung (Jan-Ole):} In welchem Maße verbessert eine
  zentralisierte und sichere Datenverwaltung die Transparenz und
  Nachvollziehbarkeit von Geschäftsreisen?
\end{itemize}

\section{Vorgehensweise und Methodik}
Die Umsetzung erfolgt iterativ auf Basis des Pflichtenhefts und der
Implementierungsprotokolle. Anforderungen werden in Module zerlegt,
technisch realisiert und anschließend begründet sowie bewertet.
Für die Evaluierung werden KPIs und qualitative Kriterien herangezogen,
die typische Reiseabläufe abbilden.
\subsection{Methodik der Modulbewertung}
Pro Modul werden Bewertungskriterien definiert (z.\,B. Zuverlässigkeit,
Effizienz, Transparenz). Die Ergebnisse werden in Kapitel 8 zusammengefasst.
\subsection{Testaufbau und Datengrundlage}
Funktionstests sichern die Hauptabläufe; Reminder- und Import-Tests
validieren die zeitkritischen Funktionen. Die Bewertung erfolgt anhand
typischer Reiseszenarien.

\section{Aufbau der Arbeit}
Kapitel 1 ist die Präambel, Kapitel 2 die Einleitung und Kapitel 3 die theoretischen
Grundlagen (Import, Karten, Benachrichtigungen, Datenverwaltung). Kapitel 4 bis 7
beschreiben die Implementierung dieser Module in Skyline. Kapitel 8 fasst Bewertung
und Evaluation zusammen. Kapitel 9 behandelt die technische Architektur sowie die
projektbezogene Umsetzung (Karten, Import, Benachrichtigungen, Datenverwaltung,
Gamification, Tests). Kapitel 10 enthält die Installation, Kapitel 11 schließt mit
Zusammenfassung und Ausblick.
