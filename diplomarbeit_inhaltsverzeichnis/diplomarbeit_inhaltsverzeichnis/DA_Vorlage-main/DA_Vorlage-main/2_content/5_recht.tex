%!TEX root = ../DA_MainDocument.tex

% ============================================================
\section{Rechtliche Rahmenbedingungen (projektrelevant)}\label{sec:recht}
% ============================================================

Die Skyline-App verarbeitet personenbezogene Daten in mehreren funktionalen
Bereichen: Benutzerprofile, Flug- und Reisedaten, Notizen, Checklisten sowie
hochgeladene Reisedokumente.
Rechtsdogmatisch handelt es sich um eine automatisierte Verarbeitung
personenbezogener Daten im Sinne des Art.~4 Z~1 und Z~2 DSGVO
\cite{gdpr}, da Informationen verarbeitet werden, die sich auf identifizierte
oder identifizierbare natürliche Personen beziehen.
Der Anwendungsbereich der Datenschutz-Grundverordnung ist damit gemäß
Art.~2 Abs.~1 DSGVO vollständig eröffnet.

Da Skyline auf Supabase als Backend-Plattform aufsetzt, werden Authentifizierung,
Datenbankzugriff und Dateispeicherung technisch zentral über eine
cloudbasierte Infrastruktur verwaltet \cite{supabasePlatform}.
Diese Architektur ermöglicht die Umsetzung datenschutzrechtlicher Anforderungen
auf mehreren Ebenen gleichzeitig: sichere Authentifizierung, datenbankbasierte
Zugriffskontrolle über Row-Level Security sowie kontrollierter Dateizugriff
über zeitlich begrenzte Signed URLs.

% ============================================================
\subsection{Datenschutzrechtliche Vorgaben (DSGVO)}
% ============================================================

Die DSGVO \cite{gdpr} bildet den primären Rechtsrahmen für die Verarbeitung
personenbezogener Daten innerhalb von Skyline.
Ergänzend gilt das österreichische Datenschutzgesetz (DSG) \cite{dsg},
das nationale Präzisierungen und Öffnungsklauseln zur DSGVO enthält.
Maßgeblich sind insbesondere die Grundsätze aus Art.~5 DSGVO:
Rechtmäßigkeit, Zweckbindung, Datenminimierung, Richtigkeit,
Speicherbegrenzung sowie Integrität und Vertraulichkeit.

Für Skyline sind vor allem die Grundsätze der Zweckbindung,
der Datenminimierung sowie der Integrität und Vertraulichkeit
technisch umzusetzen.
Reiseprofile können, auch ohne besondere Kategorien im Sinne des Art.~9 DSGVO
zu berühren, durch die Kombination von Bewegungsdaten, Aufenthaltsorten
und beruflichen Kontexten ein erhöhtes Schutzbedürfnis begründen.
Die Systemarchitektur muss daher sicherstellen, dass unbefugte Zugriffe
strukturell -- und nicht nur durch App-seitige Filter -- ausgeschlossen sind.

% ------------------------------------------------------------
\subsubsection{Speicherung und Verarbeitung personenbezogener Daten}
% ------------------------------------------------------------

In Skyline werden die folgenden Kategorien personenbezogener Daten verarbeitet:

\begin{itemize}
  \item \textbf{Stammdaten:} Name, E-Mail-Adresse (via Supabase Auth),
        optional Profilbild (\texttt{avatar\_url}), Kontotyp
        (\texttt{account\_type}: \texttt{worker}/\texttt{company}),
        nutzerspezifische Einstellungen (\texttt{preferences}, JSONB).

  \item \textbf{Reise- und Flugdaten:} Flugnummer, Airline, Datum,
        Abflug- und Ankunftszeiten (\texttt{departure\_at},
        \texttt{arrival\_at}), Gate, Terminal, Sitzplatznummer,
        Buchungsbestätigungscode, Buchungsreferenz (PNR),
        Flugstatus sowie die berechnete Flugdistanz.
        Diese Felder werden in der Tabelle \texttt{user\_flights}
        flugbezogen gespeichert.

  \item \textbf{Inhalts- und Kommunikationsdaten:} Freitextnotizen
        (\texttt{user\_notes}), Checklisten mit Einzelpunkten
        (\texttt{user\_checklists}, \texttt{user\_checklist\_items})
        sowie ein optionales Erinnerungsfeld (\texttt{reminder\_at}).

  \item \textbf{Dokumentendaten:} Dateiname, MIME-Type, Dateigröße,
        Storage-Pfad sowie der Dateiinhalt selbst (im privaten
        Storage-Bucket \texttt{flight-documents}).
        Metadaten werden in der Tabelle \texttt{flight\_documents}
        verwaltet; der Kategorie-Typ (\texttt{document\_type}) erlaubt
        eine strukturierte Unterscheidung zwischen Boardingpässen,
        Buchungsbestätigungen, Quittungen und sonstigen Dokumenten.
\end{itemize}

Die Verarbeitung erfolgt zweckgebunden im Sinne des Art.~5 Abs.~1
lit.~b DSGVO zur Organisation und Verwaltung von Flugreisen.
Die Datenbankstruktur unterstützt diese Zweckbindung unmittelbar:
Alle Zusatzinformationen (Notizen, Dokumente, Checklisten) sind über
einen Foreign Key mit \texttt{ON DELETE CASCADE} an genau einen
Flugkontext gebunden -- eine Verwahrung losgelöster Datensätze ist
strukturell ausgeschlossen.

Im Teamkontext basiert der Datenzugriff auf einer kontrollierten
Zuordnung über \texttt{company\_id} und \texttt{company\_members}.
Ohne Mitgliedschaft in einer Firma ist kein Zugriff auf Firmendaten
möglich; das Prinzip der Zugriffsbeschränkung ist damit technisch
erzwungen, nicht nur konzeptionell formuliert.

% ------------------------------------------------------------
\subsubsection{Maßnahmen zur sicheren Datenverarbeitung}
% ------------------------------------------------------------

Gemäß Art.~32 DSGVO sind geeignete technische und organisatorische
Maßnahmen (TOM) umzusetzen, um ein dem Risiko angemessenes Schutzniveau
zu gewährleisten \cite{edpbArt32}.
In Skyline werden die folgenden technischen Maßnahmen implementiert:

\begin{description}
  \item[Authentifizierung über Supabase Auth \cite{supabaseAuth}:]
    Zugriff auf personenbezogene Daten ist ausschließlich nach
    erfolgreicher Authentifizierung möglich.
    Die Benutzer-ID (\texttt{auth.uid()}) bildet die Grundlage aller
    datenbankinternen Zugriffskontrollen.
    Login-, Signup- und Passwort-Reset-Prozesse sind vollständig
    über Supabase Auth abgewickelt; Passwörter werden nicht im
    Klartext gespeichert.

  \item[Row-Level Security (RLS) \cite{supabaseRLS}:]
    Alle operativen Tabellen (\texttt{user\_flights},
    \texttt{user\_notes}, \texttt{user\_checklists},
    \texttt{user\_checklist\_items}, \texttt{flight\_documents} u.\,a.)
    sind durch RLS-Policies geschützt.
    Abfragen werden datenbanksseitig auf die eigenen Datensätze
    des authentifizierten Nutzers eingeschränkt.
    Selbst fehlerhafte oder böswillige Anfragen ohne clientseitigen
    Filter können nicht auf fremde Daten zugreifen.
    Die Policies folgen dem Muster
    \texttt{USING (profile\_id = auth.uid())} für Lese-,
    Änderungs- und Löschoperationen sowie
    \texttt{WITH CHECK (profile\_id = auth.uid())} für Schreiboperationen.

  \item[Private Storage-Buckets \cite{supabaseStorage}:]
    Dokumente werden im privaten Supabase-Storage-Bucket
    \texttt{flight-documents} abgelegt.
    Kein dauerhafter öffentlicher Direktzugriff ist möglich;
    der Bucket erfordert eine aktive Zugriffskontrolle für
    jeden Dateiaufruf.

  \item[Signed URLs mit begrenzter Gültigkeit:]
    Dateizugriffe erfolgen über zeitlich limitierte,
    signierte URLs (Gültigkeitsdauer: 3.600 Sekunden).
    Abgelaufene URLs werden vom \texttt{DocumentService}
    automatisch erneuert und in der Datenbank aktualisiert.
    Dauerhaft gültige Direktlinks auf Dokumente sind
    architektonisch ausgeschlossen.

  \item[Transportverschlüsselung:]
    Die gesamte Kommunikation zwischen App und Supabase-Backend
    erfolgt über HTTPS/TLS.
    Vertraulichkeit und Integrität der Daten sind während
    des Transports gewährleistet.

  \item[Strukturierte Speicherpfade:]
    Dokumente im Storage-Bucket werden nach dem Schema
    \texttt{\{userId\}/\{flightId\}/\{timestamp\}\_\{fileName\}}
    abgelegt.
    Dadurch ist die Eigentümerzuordnung auch auf Dateisystemebene
    eindeutig; zufällige Namenskollisionen werden durch den
    Zeitstempel verhindert.

  \item[Referenzielle Integrität und Datenlebenszyklus:]
    Foreign Keys mit \texttt{ON DELETE CASCADE} stellen sicher,
    dass beim Löschen eines Nutzerprofils oder eines Fluges
    alle verknüpften Daten (Notizen, Checklisten, Dokumente)
    automatisch mitgelöscht werden.
    Verwaiste Datensätze ohne zugehörigen Kontext entstehen
    strukturell nicht.
    Dies unterstützt die Speicherbegrenzung im Sinne des
    Art.~5 Abs.~1 lit.~e DSGVO.

  \item[Transaktionaler Rollback bei Upload-Fehlern:]
    Schlägt nach einem erfolgreichen Storage-Upload das
    Schreiben der Metadaten in die Datenbank fehl,
    löscht der \texttt{DocumentService} die hochgeladene
    Datei automatisch wieder.
    Es entstehen keine unbeschrifteten Dateien im Bucket,
    denen kein Datenbankdatensatz gegenübersteht.
\end{description}

% ------------------------------------------------------------
\subsubsection{Rechte der betroffenen Personen}
% ------------------------------------------------------------

Die DSGVO gewährt betroffenen Personen umfangreiche Rechte,
die technisch umzusetzen sind \cite{dsb}:

\begin{itemize}
  \item \textbf{Auskunftsrecht (Art.~15 DSGVO):}
        Skyline speichert alle nutzerbezogenen Daten strukturiert
        in verknüpften Tabellen; eine vollständige Datenauskunft
        kann über die \texttt{profile\_id} als Einstiegspunkt
        automatisiert erstellt werden.

  \item \textbf{Recht auf Berichtigung (Art.~16 DSGVO):}
        Profildaten (\texttt{full\_name}, \texttt{avatar\_url},
        \texttt{preferences}), Flugdaten und Notizen sind
        über die App editierbar.
        RLS stellt sicher, dass ein Nutzer nur eigene Daten
        ändern kann.

  \item \textbf{Recht auf Löschung (Art.~17 DSGVO):}
        Durch \texttt{ON DELETE CASCADE} bewirkt das Löschen
        des Nutzerprofils die vollständige Kaskadenlöschung
        aller verknüpften Datensätze.
        Storage-Objekte werden zusätzlich explizit gelöscht,
        da Storage und Datenbank getrennte Systeme sind.

  \item \textbf{Recht auf Einschränkung der Verarbeitung
        (Art.~18 DSGVO) und Datenübertragbarkeit (Art.~20 DSGVO):}
        Die strukturierte, relationale Datenhaltung bildet die
        technische Grundlage für einen Export im maschinenlesbaren
        Format.
\end{itemize}

% ============================================================
\subsection{Datenschutz in Skyline}
% ============================================================

Datenschutz wird in Skyline als architektonisches Prinzip umgesetzt
(\emph{Privacy by Design} gemäß Art.~25 DSGVO \cite{edpbArt25}).
Zugriffskontrollen werden nicht ausschließlich auf Ebene der
Benutzeroberfläche implementiert, sondern primär über
serverseitige Datenbankrichtlinien erzwungen.
Damit entspricht die Implementierung dem Grundsatz
\emph{Privacy by Default}: Standardmäßig sind alle Daten
auf den eigenen Nutzerkontext beschränkt -- eine explizite
Freigabe ist erforderlich, um Daten für andere sichtbar zu machen.

% ------------------------------------------------------------
\subsubsection{Authentifizierung und Zugriffsschutz}
% ------------------------------------------------------------

Der Zugriff auf personenbezogene Daten ist in Skyline ausschließlich
für authentifizierte Benutzer möglich \cite{supabaseAuth}.
Nicht authentifizierte Anfragen können auf keine der durch RLS
geschützten Tabellen zugreifen; der einzige öffentlich zugängliche
Bereich ist die Flughafenstammdatenbank (\texttt{airports}),
die keine personenbezogenen Daten enthält.

Supabase Auth stellt JSON Web Tokens (JWT) aus \cite{rfc7519},
die bei jeder Datenbankoperation serverseitig validiert werden.
Der darin enthaltene Wert \texttt{auth.uid()} entspricht der
\texttt{profile\_id} in allen Nutzertabellen und bildet die
einzige Grundlage für RLS-Entscheidungen.

Im Teamkontext werden zusätzlich Rollen (\texttt{owner},
\texttt{worker}) aus der Tabelle \texttt{company\_members}
berücksichtigt.
Firmeninhaber können Einladungslinks generieren
(\texttt{company\_invites}); erst nach Annahme der Einladung
entsteht ein \texttt{company\_members}-Eintrag, der den
erweiterten Datenzugriff freischaltet.

% ------------------------------------------------------------
\subsubsection{RLS-Policies in Supabase}
% ------------------------------------------------------------

Die zentrale Zugriffskontrolle in Skyline erfolgt über
Row-Level Security-Policies auf PostgreSQL-Ebene \cite{supabaseRLS}.
Tabelle~\ref{tab:rls_uebersicht} gibt eine Übersicht der
Policy-Logik für die wichtigsten Tabellen:

\begin{table}[htbp]
  \centering
  \caption{Übersicht der RLS-Policy-Logik in Skyline}
  \label{tab:rls_uebersicht}
  \begin{tabularx}{\textwidth}{l X}
    \toprule
    \textbf{Tabelle} & \textbf{Policy-Logik (vereinfacht)} \\
    \midrule
    \texttt{profiles}
      & Eigenes Profil: \texttt{id = auth.uid()} \\
      & Firmenmitglieder: EXISTS-Abfrage gegen \texttt{company\_members}
        (gleiche \texttt{company\_id}) \\
    \texttt{user\_flights}
      & \texttt{profile\_id = auth.uid()} für alle Operationen \\
    \texttt{user\_notes}
      & \texttt{profile\_id = auth.uid()} für alle Operationen \\
    \texttt{user\_checklists}
      & \texttt{profile\_id = auth.uid()} für alle Operationen \\
    \texttt{user\_checklist\_items}
      & Indirekt via \texttt{user\_checklists}: EXISTS-Abfrage,
        ob die übergeordnete Checkliste dem Nutzer gehört \\
    \texttt{flight\_documents}
      & \texttt{profile\_id = auth.uid()} für alle Operationen \\
    \texttt{airports}
      & Öffentlich lesbar (\texttt{anon} und \texttt{authenticated});
        kein Personenbezug \\
    \texttt{achievements}
      & Öffentlich lesbar; keine personenbezogenen Daten \\
    \texttt{notifications}
      & \texttt{user\_id = auth.uid()} für alle Operationen \\
    \bottomrule
  \end{tabularx}
\end{table}

Die serverseitige Durchsetzung dieser Policies reduziert das Risiko,
dass fehlerhafte oder manipulierte Client-Abfragen zu unbefugtem
Datenzugriff führen.
Damit wird das Prinzip der Integrität und Vertraulichkeit nach
Art.~5 Abs.~1 lit.~f DSGVO auf technischer Ebene abgesichert --
in Übereinstimmung mit den Empfehlungen der
OWASP Mobile Top 10 \cite{owaspMobileTop10},
die serverseitige Autorisierungsprüfung als kritische Anforderung
für mobile Applikationen ausweisen.

% ------------------------------------------------------------
\subsubsection{Dokumentenspeicherung und Dateirechte}
% ------------------------------------------------------------

Dokumente werden im privaten Supabase-Storage-Bucket
\texttt{flight-documents} gespeichert \cite{supabaseStorage}.
Metadaten (Dateiname, Typ, Größe, Storage-Pfad, Gültigkeitsdauer
der Signed URL) werden in der Tabelle \texttt{flight\_documents}
verwaltet.
Diese Trennung von Metadaten und Binärdaten ermöglicht eine
effiziente Listendarstellung ohne vollständige Dateiübertragung.

Für den Dateizugriff werden ausschließlich zeitlich begrenzte
Signed URLs verwendet.
Öffentliche Permanentlinks auf Nutzer-Dokumente sind
architektonisch ausgeschlossen.
Abgelaufene URLs werden serverseitig automatisch erneuert
und in der Datenbank aktualisiert, sodass die App stets
valide Zugriffslinks vorhält.

Löschvorgänge umfassen sowohl das Storage-Objekt als auch
den zugehörigen Metadateneintrag in der Datenbank.
Durch den transaktionalen Rollback-Mechanismus im
\texttt{DocumentService} wird zudem sichergestellt,
dass keine Dateien im Bucket verbleiben, die keinem
Datenbankdatensatz mehr zugeordnet sind -- ein zentraler
Beitrag zur Speicherbegrenzung im Sinne der DSGVO.

Das Telekommunikationsgesetz 2021 (TKG 2021) \cite{tkg2021}
ist im Kontext der Benachrichtigungs- und Erinnerungsfunktionen
von Skyline relevant, da die App lokale Push-Benachrichtigungen
über den Notification-Service ausliefert.
Die Speicherung von \texttt{fire\_at}, \texttt{kind} und
\texttt{payload} in der Tabelle \texttt{notifications} erfolgt
ausschließlich nutzerbezogen und RLS-gesichert.
