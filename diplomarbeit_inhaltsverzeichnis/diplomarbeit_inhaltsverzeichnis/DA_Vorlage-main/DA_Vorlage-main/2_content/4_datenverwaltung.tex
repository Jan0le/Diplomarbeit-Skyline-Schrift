%!TEX root = ../DA_MainDocument.tex
\chapter{Zentrale \& sichere Datenverwaltung (JanOle)}\label{chapter:datenverwaltung}
\section{Grundlagen zentraler Datenhaltung}
Zentrale Datenhaltung bedeutet eine gemeinsame Quelle fuer alle Reiseinformationen.
Dadurch wird Informationsfragmentierung reduziert und die Nachvollziehbarkeit erhoeht.
\subsection{Single Source of Truth}
Alle relevanten Reisedaten werden an einem Ort gepflegt, sodass kein
Versionskonflikt zwischen E-Mail, Dateien und Notizen entsteht.
\subsection{Problem verteilter Datenquellen}
Verteilte Daten fuehren zu doppelten Eintraegen, Suchaufwand und fehlender
Uebersicht.
\subsection{Relevanz fuer Geschaeftsreisen}
Geschaeftsreisen erfordern klare Nachweise, Belege und schnelle Auskunft
gegenueber Buchhaltung oder Management.
\section{Anforderungen an Datenverwaltung}
Die Datenverwaltung muss strukturiert, nachvollziehbar und sicher sein.
\subsection{Strukturierte Entitaeten (Flights, Notes, Docs, Checklists)}
Flights sind die zentrale Entitaet, an die Notizen, Checklisten und Dokumente
angehaengt werden.
\subsection{Transparenz \& Statusuebersicht}
Eine klare Statusanzeige zeigt, ob Informationen vorhanden oder fehlend sind.
\subsection{Nachvollziehbarkeit \& Historie}
Zeitstempel und Metadaten machen Aenderungen nachvollziehbar.
\subsection{Synchronisierung (MultiDevice)}
Daten sollen auf mehreren Geraeten konsistent verfuegbar sein.
\subsection{Rollen \& Zugriffsrechte}
Im Unternehmenskontext sind Rollen notwendig (Owner/Worker), um Zugriff zu
begrenzen.
\subsection{Datenintegritaet \& Validierung}
Validierungen sichern, dass Pflichtdaten vorhanden sind (z. B. \texttt{flight\_id}).
\section{Sicherheitskonzept}
Sicherheit basiert auf Authentifizierung und Row Level Security in Supabase
\cite{supabaseAuth,supabaseRLS}.
\subsection{Authentifizierung (Supabase Auth)}
Nutzeridentitaet wird ueber Supabase Auth verwaltet \cite{supabaseAuth}.
\subsection{RowLevel Security (RLS)}
RLS-Policies verhindern, dass Nutzer fremde Daten lesen oder aendern
\cite{supabaseRLS}.
\subsection{Policies pro Tabelle}
Jede Tabelle hat eigene Policies fuer SELECT/INSERT/UPDATE/DELETE.
\subsection{StorageSicherheit (Dokumente, Bilder)}
Dokumente werden in privaten Buckets gespeichert und ueber Policies abgesichert
\cite{supabaseStorage}.
\subsection{Datenschutzrechtliche Anforderungen}
Die DSGVO fordert Datenminimierung, Zugriffskontrolle und Loeschmoeglichkeiten
\cite{gdpr}.
\section{Implementierung in Skyline}
Die Implementierung nutzt Supabase Postgres fuer Daten und Supabase Storage
fuer Dateien \cite{supabaseStorage}.
\begin{figure}[h]
  \centering
  \fbox{\parbox{0.8\textwidth}{TODO: Datenmodell/Supabase-Tabellen oder Schema-Visualisierung einfuegen.}}
  \caption{TODO: Zentrales Datenmodell in Skyline}
  \label{fig:todo_datenmodell}
\end{figure}
\subsection{Datenmodell \& Tabellenstruktur}
Das Schema umfasst \texttt{user\_flights}, notes, checklists, documents und profiles.
\subsection{Beziehung Flight <-> Notes/Docs/Checklists}
Alle sekundaren Entitaeten verweisen ueber \texttt{flight\_id} auf den Trip.
\subsection{StorageBucket \& Signed URLs}
Dokumente werden im Storage abgelegt; Signed URLs erlauben sicheren Zugriff
\cite{supabaseStorage}.
\subsection{SyncStrategie \& Caching}
Die App synchronisiert Daten bei App-Start und nutzt lokales Caching fuer
Offline-Zugriff.
\subsection{Fehlerfaelle \& Wiederherstellung}
Fehler werden abgefangen; Daten bleiben zentral gesichert und koennen
wieder geladen werden.
\section{Bewertung der Wirkung}
Bewertet wird, ob Transparenz und Nachvollziehbarkeit messbar steigen.
\subsection{TransparenzKPI}
KPI: Zeit, um den Status einer Reise zu verstehen, und Anteil vollstaendiger
Datensaetze.
\subsection{NachvollziehbarkeitKPI}
KPI: Zeit bis ein Dokument wiedergefunden wird und Anteil fehlender Belege.
\subsection{Vergleich zu dezentralen Loesungen}
Zentralisierte Ablage reduziert Rueckfragen und manuellen Suchaufwand.
\section{Ergebnis}
Die zentrale Datenverwaltung verbessert Transparenz und Nachvollziehbarkeit
insbesondere in Geschaeftsreisen.
\subsection{Verbesserungen in Transparenz}
Status und Dokumente sind jederzeit sichtbar.
\subsection{Verbesserungen in Nachvollziehbarkeit}
Historie und Metadaten ermoeglichen klare Rueckverfolgung.
\subsection{Schlussfolgerung}
Single Source of Truth kombiniert mit Security-by-Design liefert nachhaltigen
Mehrwert fuer Organisation und Nutzer.
