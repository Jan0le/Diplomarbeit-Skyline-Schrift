%!TEX root = ../DA_MainDocument.tex
\chapter{Bewertung und Evaluation}\label{chapter:evaluation}

\section{Überblick der Bewertungskriterien}
Die Evaluierung der Module erfolgte anhand einheitlicher Kriterien: Zuverlässigkeit,
Effizienz, Transparenz und Nachvollziehbarkeit. KPIs und qualitative Kriterien wurden
pro Modul angewendet, um die Wirkung messbar zu machen.

\section{Beantwortung der Forschungsfragen}
Die Arbeit wurde von vier Forschungsfragen geleitet (Kapitel 2). Die Evaluation
beantwortet diese wie folgt:

\textbf{Forschungsfrage 1 – Weltkarte} (\glqq{}Wie sehr steigert eine visuelle
Darstellung die Übersicht und Transparenz bei Vielreisenden?\grqq{}): Kapitel
\ref{chapter:weltkarte} dokumentiert die Umsetzung der interaktiven Kartenvisualisierung.
Die Bewertung zeigt, dass die geografische Darstellung von Flugrouten die Orientierung
erleichtert und eine schnelle Übersicht über Reisehistorie und geplante Flüge liefert.
Für Vielreisende steigert dies messbar die Übersicht und Transparenz.

\textbf{Forschungsfrage 2 – Import} (\glqq{}Wie viel Zeitersparnis bringt die
automatische Datenübernahme im Vergleich zur manuellen Eingabe?\grqq{}): Kapitel
\ref{chapter:import} beschreibt die Implementierung von QR-Scan, OCR und E-Mail-Import.
Die automatische Erfassung reduziert den manuellen Aufwand erheblich; Nutzer können
Flüge in Sekunden statt Minuten erfassen. Die Bewertung dokumentiert die
Zeitersparnis gegenüber manueller Eingabe.

\textbf{Forschungsfrage 3 – Benachrichtigungen} (\glqq{}Wie sehr erhöhen proaktive
Benachrichtigungen die Zuverlässigkeit und Effizienz bei der Reiseorganisation?\grqq{}):
Die Implementierung in Kapitel \ref{chapter:notifications} zeigt, dass Reminder-Offsets
(Check-in, Boarding, fehlende Unterlagen) kritische Fehlzustände reduzieren. Die
Bewertung deutet auf eine messbare Steigerung von Zuverlässigkeit und Effizienz hin.

\textbf{Forschungsfrage 4 – Datenverwaltung} (\glqq{}In welchem Maße verbessert eine
zentralisierte und sichere Datenverwaltung die Transparenz und Nachvollziehbarkeit
von Geschäftsreisen?\grqq{}): Kapitel \ref{chapter:datenverwaltung} dokumentiert die
Umsetzung. Die zentrale Ablage, RLS-Sicherheit und strukturierte Entitäten führen
zu besserer Transparenz (Status sichtbar) und Nachvollziehbarkeit (Dokumente
auffindbar, Historie vorhanden).

\section{Zusammenfassung der Modulbewertungen}
\subsection{Import und Datenverwaltung}
Kapitel \ref{chapter:import} und \ref{chapter:datenverwaltung} dokumentieren die
Implementierung und Bewertung der Datenaufnahme sowie der zentralen Speicherung. Die Bewertung erfolgte über
Importqualität, Zeitaufwand und Transparenz-KPIs.

\subsection{Kartenvisualisierung und Benachrichtigungen}
Kapitel \ref{chapter:weltkarte} und \ref{chapter:notifications} dokumentieren die
Implementierung und Bewertung der visuellen Darstellung sowie der proaktiven Erinnerungsfunktion. Bewertet wurden
Performance, Nutzerakzeptanz und Zuverlässigkeit der Reminder.

\section{Gesamtbewertung}
Die Kombination aus Import, Datenverwaltung, Karte und Benachrichtigungen
liefert einen messbaren Mehrwert für die Reiseorganisation. Die empirischen
Ergebnisse sind in den jeweiligen Modulkapiteln dokumentiert. Die
Forschungsfragen können auf Basis der Modulbewertungen positiv beantwortet
werden: Die visuelle Weltkarte steigert Übersicht und Transparenz; der automatische
Import bringt messbare Zeitersparnis; proaktive Benachrichtigungen erhöhen
Zuverlässigkeit und Effizienz; die zentrale Datenverwaltung verbessert Transparenz
und Nachvollziehbarkeit von Geschäftsreisen.
