%!TEX root = ../DA_MainDocument.tex
\chapter{Implementierung: Automatischer Import von Boardkarten \& Buchungsdaten (Boris)}\label{chapter:import}

Dieses Kapitel beschreibt die Umsetzung des automatischen Imports in Skyline. Die theoretischen Grundlagen (Datenquellen, Validierung, Fehlerbehandlung) werden in Kapitel 3 behandelt.

\section{Implementierung in Skyline}
In Skyline werden Flugdetails aus QR-Codes, Bildern und Dokumenten extrahiert und in die Flugverwaltung übernommen. Die App kombiniert mehrere Quellen, um die Abdeckung zu erhöhen und die Wahrscheinlichkeit eines erfolgreichen Imports zu steigern.
\begin{figure}[h]
  \centering
  \fbox{\parbox{0.8\textwidth}{TODO: Screenshot des Import-Flows (QR/OCR/E-Mail) einfuegen.}}
  \caption{TODO: Import-Flow in Skyline}
  \label{fig:todo_import_flow}
\end{figure}

\subsection{QR-Scan und BCBP-Parsing}
Skyline nutzt die Kamera des Geräts zum Scannen von Boarding-Pass-QR-Codes. Die BCBP-Struktur (IATA-Standard) wird geparst, sodass Airline, Flugnummer, Datum sowie Abflug- und Zielairport extrahiert werden. Nicht alle Boardingpässe sind standardkonform; Skyline fängt Abweichungen ab und ermöglicht bei Bedarf manuelle Ergänzung.

\subsection{OCR-Verarbeitung aus Bildern und PDFs}
Wenn kein QR-Code vorhanden ist, kann ein Foto des Boardingpasses oder ein PDF-Dokument importiert werden. Ein OCR-Service extrahiert den Text; Skyline parst anschließend relevante Felder (IATA-Codes, Datum, Flugnummer) über heuristische Regeln. Schriftgröße, Kontrast und Bildqualität beeinflussen die Erkennungsrate; die Validierungsschicht gleicht OCR-Fehler teilweise aus.

\subsection{E-Mail-Import und Buchungsdaten}
Buchungsbestätigungen können per E-Mail-Import in Skyline übernommen werden. Skyline unterstützt typische Formate gängiger Airlines (strukturierte Felder wie Buchungsreferenz, Routing, Zeiten). Unterschiedliche E-Mail-Templates erfordern flexible Parser; die Implementierung deckt die häufigsten Muster ab und erweitert diese iterativ.

\subsection{Validierung und Plausibilitätsprüfung}
Alle importierten Daten durchlaufen eine Validierungsstufe. IATA-Airport-Codes werden gegen eine Referenz geprüft, Datumsformate normalisiert und offensichtlich ungültige Werte abgefangen. Plausibilitätschecks verhindern fehlerhafte Einträge in der Flugverwaltung; unvollständige Daten werden gekennzeichnet und können manuell ergänzt werden.

\subsection{Feld-Mapping und Datennormalisierung}
Externe Datenquellen liefern unterschiedliche Feldnamen und Formate. Skyline mappt diese auf das einheitliche interne Schema (Abflug, Ankunft, Datum, Airline, Flugnummer usw.). Datums- und Zeitformate werden normalisiert; Mehrsprachigkeit und Zeitzonen werden berücksichtigt.

\subsection{Deduplizierung}
Mehrfachimporte desselben Flugs (z.\,B. QR + E-Mail) werden erkannt. Skyline vergleicht Flugnummer, Datum und Airports; bestehende Einträge werden aktualisiert statt Duplikate anzulegen. Der Nutzer erhält eine Rückmeldung, ob der Import neu oder als Update gewertet wurde.

\subsection{Fehlerbehandlung und Nutzerfeedback}
Fehlgeschlagene Imports (z.\,B. unleserlicher QR-Code, OCR-Fehler) werden nutzerfreundlich kommuniziert. Skyline bietet Fallback auf manuelle Erfassung und zeigt konkrete Hinweise (z.\,B. „Airport-Code unbekannt“). Asynchrone Verarbeitung hält die Oberfläche responsiv; Fortschritt und Ergebnis werden angezeigt.

\subsection{Anbindung an die Flugverwaltung}
Importierte Flüge werden direkt in die zentrale Flugverwaltung (Kapitel 5) übernommen. Nach erfolgreichem Import können Nutzer sofort Notizen, Dokumente oder Erinnerungen verknüpfen; die Weltkarte (Kapitel 6) und Benachrichtigungen (Kapitel 7) greifen auf dieselben Daten zu.

\section{Ergebnis}
Der automatische Import ist ein zentraler Mehrwert von Skyline und entlastet Nutzer im Reisealltag.
\subsection{Bewertung der Importqualität}
Die Qualität zeigt sich in hoher Trefferquote bei QR- und E-Mail-Import sowie akzeptabler Erkennungsrate bei OCR. Geringe Nachbearbeitung und seltene manuelle Korrekturen bestätigen die Wirksamkeit von Validierung und Feld-Mapping.
\subsection{Nutzen im Reise-Workflow}
Nutzer können Flüge schneller erfassen, früher mit Planung starten und Belege direkt zuordnen. Die Kombination aus mehreren Datenquellen erhöht die Abdeckung auch bei heterogenen Buchungsabläufen.
\subsection{Grenzen und Verbesserungspotenzial}
Nicht alle Airlines und E-Mail-Formate sind abgedeckt; OCR ist von Bildqualität abhängig. Für die Zukunft sind robustere Parser, erweiterte Airline-Templates und ggf. API-Anbindungen (z.\,B. GDS) denkbar.
