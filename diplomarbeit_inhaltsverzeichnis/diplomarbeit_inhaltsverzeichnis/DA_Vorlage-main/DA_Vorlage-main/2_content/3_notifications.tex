%!TEX root = ../DA_MainDocument.tex
\chapter{Implementierung: Benachrichtigungs- und Erinnerungsmodul (JanOle)}\label{chapter:notifications}

Dieses Kapitel beschreibt die Umsetzung der Benachrichtigungen in Skyline. Die theoretischen Grundlagen (proaktive Systeme, Notification Fatigue, Anforderungen im Reise-Kontext, Reminder-Typen) werden in Kapitel 3 behandelt.

\section{Implementierung in Skyline}
Die Implementierung nutzt Expo Notifications und eine eigene Registry für Persistenz \cite{expoNotifications}.
\subsection{ReminderOffsets \& SchedulingFlow}
Beim Speichern eines Flugs werden die Offsets geprüft und geplant. Standard-Offsets wie T-24\,h (Check-in), T-60\,m/T-30\,m (Boarding), T-12\,h (Missing Docs) und T+2\,h (Receipt) werden gemäß den in Kapitel 3 beschriebenen Heuristiken gesetzt.
\subsection{Integration beim FlightSave}
Die Scheduling-Logik wird automatisch beim Flug-Speichern angestossen.
\subsection{Cancel/Reschedule bei Updates}
Bei Änderungen werden alte Reminders gecancelt und neu gesetzt.
\subsection{PushIntegration (EAS / Expo Tokens)}
Push-Benachrichtigungen sind konzeptionell vorbereitet; für Produktion braucht es FCM/APNs via EAS.

\section{Bewertung der Wirkung}
Die Wirkung wird über Zuverlässigkeit, Effizienz und Nutzerakzeptanz bewertet.
\subsection{Zuverlässigkeit als KPI}
KPI: Anteil der Flüge ohne kritische Fehlzustände (z.\,B. fehlende Unterlagen).
\subsection{Effizienz als KPI}
KPI: Reduktion der Suchzeit nach Dokumenten und Anzahl manueller Schritte.
\subsection{Nutzerakzeptanz}
Akzeptanz wird über qualitative Rückmeldungen und Settings-Nutzung beurteilt.

\section{Ergebnis}
Das Modul erhöht die Verlässlichkeit der Reiseorganisation, wenn Timing und Relevanz stimmen.
\subsection{Reduktion kritischer Fehlzustände}
Hinweise auf Check-in und fehlende Dokumente reduzieren Fehler.
\subsection{Effizienzsteigerung}
Weniger Suchaufwand und klarere Ablaufe steigern die Effizienz.
\subsection{Gesamtbewertung}
Die Kombination aus Reminder-Offsets, Quiet Hours und Deeplinks liefert einen messbaren Mehrwert.
