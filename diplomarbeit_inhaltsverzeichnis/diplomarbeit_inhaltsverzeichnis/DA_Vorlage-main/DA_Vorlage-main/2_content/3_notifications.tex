%!TEX root = ../DA_MainDocument.tex
\chapter{Benachrichtigungs- und Erinnerungsmodul}\label{chapter:notifications}

\section{Grundlagen proaktiver Benachrichtigungssysteme}
Proaktive Benachrichtigungssysteme informieren Nutzerinnen und Nutzer nicht erst,
wenn sie aktiv nach Informationen suchen, sondern liefern relevante Hinweise
automatisch aus (\glqq Push\grqq{} statt \glqq Pull\grqq{}). Dadurch werden
Informationen genau dann verfuegbar, wenn sie benoetigt werden, ohne dass die App
geoeffnet oder ein bestimmter Bildschirm aufgerufen werden muss. In der Forschung
zu intelligenten Benachrichtigungssystemen wird diese Push-basierte Zustellung
explizit dem Paradigma gegenuebergestellt, bei dem Nutzerinnen und Nutzer
Informationen selbst abrufen muessen \cite{mehrotra2018intelligent}.

Im Kontext der Reiseorganisation ist diese Logik besonders passend, weil viele
Schritte an feste Zeitfenster gebunden sind (z. B. Online-Check-in, rechtzeitige
Anwesenheit am Gate oder \glqq Last Boarding\grqq{}). Airlines veroeffentlichen
konkrete Boarding-Zeitfenster und schliessen Boarding typischerweise einige
Minuten vor Abflug, sodass Timing zu einem objektiven Erfolgsfaktor wird (z. B.
Boarding-Beginn 60--25 Minuten vor Abflug und Boarding-Ende rund 15 Minuten vor
planmaessigem Abflug \cite{klmBoarding}). Ein proaktives System kann diese
Zeitfenster als Ausloeser nutzen, um organisatorische Fehler (Vergessen,
falsches Timing) zu reduzieren und die kognitive Belastung von Nutzerinnen und
Nutzern zu senken \cite{mehrotra2015contentdriven,mehrotra2016myphone}.

\subsection{Begriff „proaktiv“ und Abgrenzung zu reaktiven Systemen}
In der Informatikforschung wird \glqq proaktiv\grqq{} so definiert, dass ein
System auf eigene Initiative und im Sinne der Nutzerin bzw. des Nutzers handelt,
anstatt ausschliesslich auf explizite Eingaben zu reagieren. Tennenhouse
beschreibt Proactive Computing als Abkehr von klassischer, interaktiver
Nutzung (\glqq human-in-the-loop\grqq{}) hin zu Modi, in denen Menschen
\glqq above the loop\grqq{} sind und Systeme staerker autonom agieren
\cite{tennenhouse2000proactive}. Coronado und Zampunieris betonen aehnlich, dass
proaktive Systeme kontinuierlich handeln und Ereignisse antizipieren, statt nur
auf Benutzeraktionen zu warten \cite{coronado2010continuous}.

Reaktive Systeme geben Informationen dagegen primaer auf Nachfrage aus: Der
Nutzer muss selbst regelmaessig pruefen, ob es etwas zu tun gibt (z. B.
\glqq Ist der Check-in schon offen?\grqq{}). Gerade bei zeitkritischen
Reiseprozessen erhoeht eine reine Abruf-Logik die Wahrscheinlichkeit von
Versaeumnissen, weil Fristen nur dann beachtet werden, wenn sie aktiv im Blick
gehalten werden. Proaktive Systeme verlagern die Initiierung geeigneter Hinweise
teilweise vom Menschen auf das System und koennen so typische Fehlerklassen
(\glqq zu spaet bemerkt\grqq{}, \glqq vergessen\grqq{}) verringern
\cite{mehrotra2018intelligent}.

\subsection{Erinnerungslasten in der Reiseorganisation}
Reiseablaeufe lassen sich psychologisch dem Bereich der Prospective Memory
(prospektives Gedaechtnis) zuordnen: Man muss sich daran erinnern, zu einem
spaeteren Zeitpunkt oder bei einem bestimmten Ereignis eine Handlung
auszufuehren (z. B. Online-Check-in, Dokumente bereit legen, Belege sichern).
Systematische Uebersichten und Meta-Analysen zeigen, dass solche
Alltagsleistungen durch externe Gedaechtnishilfen messbar verbessert werden
koennen \cite{jones2021prospectivememory,ball2024reminders}. Aktuelle Arbeiten
fassen das bewusste Auslagern von Absichten an externe Werkzeuge als
\glqq Intention Offloading\grqq{} zusammen, etwa in Form von Notizen, Kalendern
oder digitalen Remindern \cite{gilbert2023intentionoffloading}.

Fuer das Design von Skyline ist das relevant, weil viele Reiseaufgaben genau
diesem Muster folgen: Nutzerinnen und Nutzer muessen sich spaeter an klare
To-dos erinnern, waehrend sie parallel anderen Taetigkeiten nachgehen.
Reminders koennen die sogenannte Monitoring-Anforderung senken, also das
staendige \glqq im Blick behalten\grqq{} eines zukuenftigen Zeitpunkts oder
Ereignisses. Studien zeigen zudem, dass Erinnerungen effektiver sind, wenn sie
nicht nur ein Zielereignis (\glqq Check-in-Zeit\grqq{}) nennen, sondern eine
konkrete, unmittelbar ausfuehrbare Handlung unterstuetzen (z. B. \glqq Jetzt
Online-Check-in starten\grqq{}) \cite{jones2021prospectivememory,gilbert2023intentionoffloading}.

\subsection{Notification Fatigue (Überlastung)}
Ein zentrales Risiko proaktiver Benachrichtigungssysteme ist die sogenannte
„Notification Fatigue“: Werden zu viele Hinweise gesendet, steigt die
Wahrscheinlichkeit, dass Nutzer Benachrichtigungen ignorieren oder die Funktion
komplett deaktivieren. Feldstudien zeigen einerseits eine hohe taegliche
Benachrichtigungszahl (z. B. im Mittel rund 63{,}5 Benachrichtigungen pro Tag)
und andererseits Zusammenhaenge zwischen hoeherem Notification-Volumen und
negativeren Emotionen \cite{pielot2014notifications}. Grossskalige Analysen
stuetzen dieses Bild: Sahami Shirazi et al.\ untersuchen nahezu 200 Millionen
Benachrichtigungen und zeigen, dass Nutzerinnen und Nutzer Notifications stark
nach Quelle und Typ bewerten (z. B. hohe Relevanz fuer Messaging, deutlich
geringere Interaktionsraten fuer viele andere Kategorien)
\cite{sahami2014largescale}. Eine weitere Log-Studie mit rund 794\,525
Benachrichtigungen berichtet einen Median von 56 Notifications pro Tag und eine
sehr geringe Conversion fuer viele nicht-Messaging-Hinweise
\cite{mehrotra2016myphone}.

Dass Benachrichtigungen spuerbare Kosten erzeugen koennen, wird auch
experimentell belegt. In der \glqq Do Not Disturb Challenge\grqq{} (24 Stunden
ohne Push-Notifications) fuehlten sich Teilnehmende weniger abgelenkt und
produktiver, berichteten aber gleichzeitig mehr Sorge, nicht wie erwartet
reagieren zu koennen \cite{pielot2017productive}. Arbeitspsychologische
Feldexperimente zeigen zusaetzlich, dass eine Reduktion von
notificationsbedingten Unterbrechungen positive Effekte auf Leistung und
Belastung haben kann \cite{ohly2023interruptions}.

Fuer Skyline lassen sich daraus drei Governance-Prinzipien ableiten:
\begin{itemize}
  \item \textbf{Relevanz vor Vollstaendigkeit:} Jede zusaetzliche, wenig
  relevante Notification erhoeht das Risiko von Ignorieren oder Abschalten und
  reduziert die wahrgenommene Relevanz
  \cite{pielot2014notifications,sahami2014largescale}.
  \item \textbf{Nutzerkontrolle ist zwingend:} Betriebssysteme wie Android
  stellen explizite Kontrollmechanismen bereit. Notification-Channels besitzen
  eine Importance, die kanalabhaengig definiert wird; Nutzerinnen und Nutzer
  koennen diese Importance in den Systemeinstellungen anpassen. Nach dem
  Erstellen eines Channels kann die App das Verhalten nicht mehr eigenmaechtig
  aendern (\glqq the user has complete control\grqq{})
  \cite{androidNotifications}.
  \item \textbf{Permission-Requests brauchen Kontext:} Apple betont in
  Entwickler-Schulungsmaterialien, dass die Zustimmung zu
  Benachrichtigungs-Berechtigungen steigt, wenn diese nicht direkt beim
  App-Start, sondern kontextbezogen nach einer passenden Nutzeraktion
  angefragt werden \cite{applePushPrimer2020}.
\end{itemize}

\subsection{Timing-Strategien in mobilen Apps}
Studien zu \glqq Interruptibility\grqq{} und \glqq Receptivity\grqq{} zeigen,
dass Zustellzeitpunkte nicht beliebig sind. Selbst inhaltlich sinnvolle Hinweise
koennen als stoerend empfunden werden, wenn sie in unpassenden Momenten
eintreffen, etwa waehrend komplexer Aufgaben oder intensiver sozialer
Interaktionen \cite{mehrotra2016myphone,mehrotra2018intelligent}. Ein
verbreiteter Ansatz besteht darin, \glqq opportune moments\grqq{} zu erkennen,
also geeignete Zeitfenster, und Zustellung dorthin zu verschieben. Fischer
et al.\ untersuchen Episoden mobiler Aktivitaet und argumentieren, dass
natuerliche Uebergaenge (\glqq Breakpoints\grqq{}) besonders geeignete
Zeitpunkte fuer Unterbrechungen sind \cite{fischer2011investigating}. Mehrotra
et al.\ greifen dieses Prinzip in einem \glqq defer-to-breakpoint\grqq{}-Ansatz
auf, bei dem Benachrichtigungen gezielt bis zum Ende einer Episode
aufgeschoben werden \cite{mehrotra2016myphone}.

Im Flugreisebereich bietet sich eine Kombination aus solchen HCI-Prinzipien mit
ohnehin klar strukturierten Zeitpunkten an: Online-Check-in-Fenster,
Boarding-Beginn, Boarding-Ende und Ankunft liefern natuerliche \glqq Anker\grqq{},
an denen Benachrichtigungen einerseits relevant sind und andererseits weniger
stark mit typischen Alltagsaufgaben kollidieren.

\section{Anforderungen im Reise Kontext}
Im Reise-Kontext sind Benachrichtigungen besonders zeitkritisch und muessen
an reale Ankunfts- und Abflugszeiten gekoppelt werden. In der Praxis variieren
Check-in- und Boarding-Fenster je nach Airline und Abflugort: British Airways
erlaubt den Online-Check-in ab 24 Stunden vor Abflug \cite{britishAirwaysCheckin},
Lufthansa spricht von einem Online-Check-in bis zu 30 Stunden vor Abflug, wobei
bestimmte automatisierte Check-in-Einladungen weiterhin rund 23 Stunden vorher
versendet werden \cite{lufthansaCheckin}. Austrian Airlines oeffnet den
Online-Check-in meist 47 Stunden vor Abflug von und nach Wien
\cite{austrianCheckin}, waehrend der Flughafen Wien allgemein darauf hinweist,
dass Online-Check-in je nach Airline bis zu 48 Stunden vor Abflug moeglich ist
\cite{viennaAirportOnlineCheckin}. Boarding-Zeiten sind aehnlich strukturiert;
KLM nennt einen Boarding-Beginn zwischen 60 und 25 Minuten vor Abflug und ein
Boarding-Ende etwa 15 Minuten vor planmaessigem Abflug \cite{klmBoarding}.

Fuer Skyline bedeutet das: Zeitregeln muessen als Heuristiken formuliert werden
(\glqq typischerweise\grqq{}, \glqq bei vielen Airlines\grqq{}), da sie nicht
fuer alle Fluggesellschaften und Routen exakt gelten. Eine vollautomatische
Echtzeit-Aktualisierung von Verspaetungen oder Gate-Aenderungen waere nur mit
zusaetzlichen Flugdaten-APIs realistisch, die haeufig in kommerziellen Modellen
angeboten werden. Beispiele sind FlightAware mit usage-basiertem AeroAPI-Pricing
\cite{flightawareAeroAPI}, Cirium mit Flex-APIs und Evaluation-Accounts
\cite{ciriumFlexAPIs} oder aviationstack mit einem kostenlosen Plan und
kostenpflichtigen Abos \cite{aviationstack}. Fuer ein Schulprojekt ist dies
typischerweise aufgrund von Kosten und Limits nur eingeschraenkt umsetzbar.

\subsection{Checkin Reminder (T24h)}
Viele Airlines oeffnen den Online-Check-in in einem Bereich von 24 bis
48 Stunden vor Abflug. British Airways erlaubt beispielsweise den Check-in
ab 24 Stunden vor der geplanten Abflugzeit \cite{britishAirwaysCheckin},
Lufthansa kommuniziert automatisierte Check-in-Prozesse rund 23 Stunden vor
Abflug bei einer technischen Oeffnung bis 30 Stunden \cite{lufthansaCheckin}.
Austrian Airlines nennt meist 47 Stunden vor Abflug von und nach Wien
\cite{austrianCheckin}, der Flughafen Wien erwaehnt allgemein Online-Check-in
bis zu 48 Stunden je nach Airline \cite{viennaAirportOnlineCheckin}.

Vor diesem Hintergrund ist ein Check-in-Reminder etwa 24 Stunden vor Abflug als
breit kompatible Standardheuristik sinnvoll: Er liegt innerhalb oder nahe der
typischen Oeffnungsfenster vieler Airlines und gibt ausreichend Vorlauf, um den
Check-in durchzufuehren, Sitzplaetze zu waehlen und Reiseunterlagen zu
pruefen. In Skyline wird dieser Zeitpunkt deshalb als Default gewaehlt.
\begin{figure}[h]
  \centering
  \fbox{\parbox{0.8\textwidth}{Platzhalter: Screenshot der Check-in Benachrichtigung (T-24h).}}
  \caption{Beispiel: Check-in Reminder}
  \label{fig:reminder_checkin}
\end{figure}

\subsection{Boarding Reminder (T60m / T30m)}
Boarding-Hinweise kurz vor dem Einsteigen sind besonders wirkungsvoll, da sie
in ein enges, realweltliches Zeitfenster fallen und direkt mit klaren
Fehlerzustaenden (z. B. Boarding verpasst) verknuepft sind. KLM gibt an, dass
Boarding zwischen 60 und 25 Minuten vor Abflug startet und rund 15 Minuten vor
planmaessigem Abflug schliesst \cite{klmBoarding}; aehnliche Zeitraeume finden
sich bei anderen Airlines.

Ein zweistufiger Reminder bei T-60 Minuten und T-30 Minuten bildet diese Praxis
gut ab: Der fruehere Hinweis ermoeglicht es, rechtzeitig in Richtung Gate
aufzubrechen und letzte Wege einzuplanen, waehrend der spaetere Reminder als
\glqq letzte sichere Erinnerung\grqq{} kurz vor Schliessen der Tueren fungiert.
Aus HCI-Sicht entspricht dies dem Prinzip, zeitkritische Benachrichtigungen eng
an das relevante Ereignis zu koppeln und dennoch einen ausreichenden
Handlungsspielraum zu lassen \cite{mehrotra2018intelligent,fischer2011investigating}.
\begin{figure}[h]
  \centering
  \fbox{\parbox{0.8\textwidth}{Platzhalter: Screenshot der Boarding Benachrichtigung (T-60m / T-30m).}}
  \caption{Beispiel: Boarding Reminder}
  \label{fig:reminder_boarding}
\end{figure}

\subsection{Missing Docs Reminder (T12h)}
Fehlende Unterlagen sollen möglichst früh erkannt werden, damit Nutzer noch
handeln koennen. Der Reminder wird deshalb in Skyline ca. 12 Stunden vor Abflug
gesendet, sofern wichtige Dokumente (z. B. Boardingpass, Buchungsbestaetigung
oder Rechnung) fehlen. Dadurch bleibt genug Zeit, die Unterlagen nachzureichen.

Der Zeitpunkt T-12h ist als praxisnahe Heuristik plausibel: Er liegt in der
Regel vor der unmittelbaren Reisehektik, aber nah genug am Ereignis, um als
relevant wahrgenommen zu werden. Aus der Literatur zu prospektivem Gedaechtnis
und Intention Offloading laesst sich ableiten, dass Erinnerungen besonders dann
hilfreich sind, wenn sie auf eine konkrete, zeitnahe Handlung verweisen und
noch echte Handlungsoptionen lassen
\cite{jones2021prospectivememory,gilbert2023intentionoffloading}.
\begin{figure}[h]
  \centering
  \fbox{\parbox{0.8\textwidth}{Platzhalter: Screenshot der Dokumente-Fehlen Benachrichtigung (T-12h).}}
  \caption{Beispiel: Missing Docs Reminder}
  \label{fig:reminder_missing_docs}
\end{figure}

\subsection{Receipt Reminder (T+2h)}
Nach der Ankunft wird ein Reminder für Belege gesetzt, um die Abrechnung
zu unterstuetzen. Der zeitliche Abstand von etwa zwei Stunden ist bewusst
gewaehlt, damit der Nutzer zuerst den unmittelbaren Reiseabschluss erledigen
kann (Aussteigen, Gepaeck, Transfer) und danach ruhig die Belege hochlaedt oder
fotografiert.

Aus Sicht der Interruptibility-Forschung ist dies konsistent mit dem Prinzip,
Notifications an natuerliche Uebergaenge anzuknuepfen -- in diesem Fall an den
Uebergang von der Reisephase zur Nachbereitung
\cite{fischer2011investigating,mehrotra2018intelligent}.
\begin{figure}[h]
  \centering
  \fbox{\parbox{0.8\textwidth}{Platzhalter: Screenshot der Beleg-Erinnerung (T+2h).}}
  \caption{Beispiel: Receipt Reminder}
  \label{fig:reminder_receipt}
\end{figure}

\subsection{Kontextabhängige Hinweise (z. B. fehlende Dokumente)}
Hinweise werden nur gesendet, wenn die Kontextbedingungen stimmen. Ein Beispiel
ist der Fall, dass wichtige Dokumente fehlen und der Abflug innerhalb der
naechsten 48 Stunden liegt. Dadurch werden Benachrichtigungen auf relevante
Situationen beschraenkt und die Nutzer nicht mit irrelevanten Meldungen
ueberlastet.

Formal wird Kontext im Sinne von Dey als \glqq jede Information, die genutzt
werden kann, um die Situation einer Entitaet zu charakterisieren\grqq{}
verstanden, wobei Entitaeten Personen, Orte oder Objekte sind, die fuer die
Interaktion relevant sind \cite{dey2001context}. Ein System gilt als
kontextbewusst, wenn es solchen Kontext nutzt, um relevante Informationen oder
Services bereitzustellen. Intelligente Benachrichtigungssysteme verfolgen
genau dieses Ziel: Zustellung zur \glqq richtigen Zeit\grqq{} und im
\glqq richtigen Kontext\grqq{} zu optimieren, indem sowohl Rezeptivitaet als
auch Unterbrechungskosten beruecksichtigt werden \cite{mehrotra2018intelligent}.
\begin{figure}[h]
  \centering
  \fbox{\parbox{0.8\textwidth}{Platzhalter: Screenshot eines kontextabhängigen Hinweises.}}
  \caption{Beispiel: Kontextabhängige Benachrichtigung}
  \label{fig:reminder_context}
\end{figure}

\subsection{Triggerberechnung aus departureAt/arrivalAt}
Triggerzeiten werden aus Abflug- und Ankunftszeiten abgeleitet und als Offsets
berechnet.
\subsection{Speicherung \& Persistenz (local + server)}
Lokale IDs werden gespeichert; zusaetzlich werden geplante Reminders serverseitig
persistiert, um Rescheduling zu ermoeglichen.
\subsection{Rescheduling nach AppStart}
Beim App-Start werden ausstehende Reminders geladen und neu geplant.
\subsection{DeepLinks zu Trip Details}
Benachrichtigungen enthalten Deeplinks, die direkt in die Flugdetails fuehren.
\subsection{Fehlerhandling (fehlende Zeiten, invalid data)}
Fehlende oder ungueltige Zeiten verhindern Scheduling und werden abgefangen.
\subsection{DebugAnsicht (Pending Notifications)}
Eine Debug-Ansicht zeigt geplante und serverseitige Notifications fuer Tests.
\section{Implementierung in Skyline}
Die Implementierung nutzt Expo Notifications und eine eigene Registry fuer
Persistenz \cite{expoNotifications}.
\subsection{ReminderOffsets \& SchedulingFlow}
Beim Speichern eines Flugs werden die Offsets geprueft und geplant.
\subsection{Integration beim FlightSave}
Die Scheduling-Logik wird automatisch beim Flug-Speichern angestossen.
\subsection{Cancel/Reschedule bei Updates}
Bei Aenderungen werden alte Reminders gecancelt und neu gesetzt.
\subsection{PushIntegration (EAS / Expo Tokens)}
Push-Benachrichtigungen sind konzeptionell vorbereitet; fuer Produktion
braucht es FCM/APNs via EAS.
\section{Bewertung der Wirkung}
Die Wirkung wird ueber Zuverlaessigkeit, Effizienz und Nutzerakzeptanz bewertet.
\subsection{Zuverlaessigkeit als KPI}
KPI: Anteil der Fluege ohne kritische Fehlzustaende (z. B. fehlende Unterlagen).
\subsection{Effizienz als KPI}
KPI: Reduktion der Suchzeit nach Dokumenten und Anzahl manueller Schritte.
\subsection{Nutzerakzeptanz}
Akzeptanz wird ueber qualitative Rueckmeldungen und Settings-Nutzung beurteilt.
\section{Ergebnis}
Das Modul erhoeht die Verlaesslichkeit der Reiseorganisation, wenn Timing und
Relevanz stimmen.
\subsection{Reduktion kritischer Fehlzustaende}
Hinweise auf Check-in und fehlende Dokumente reduzieren Fehler.
\subsection{Effizienzsteigerung}
Weniger Suchaufwand und klarere Ablaufe steigern die Effizienz.
\subsection{Gesamtbewertung}
Die Kombination aus Reminder-Offsets, Quiet Hours und Deeplinks liefert einen
messbaren Mehrwert.
