%!TEX root = ../DA_MainDocument.tex
\section{Projektbezogene Umsetzung}\label{sec:projektumsetzung}
Skyline wurde als \textbf{native iOS-App} entwickelt und ist im \textbf{Apple App Store}
verfügbar. Alle Module wurden primär auf iOS implementiert und getestet; eine
Android-Variante existiert, wurde jedoch im Projektzeitraum nur eingeschränkt validiert.

\subsection{Umsetzung der Karten-Visualisierung}
Die Kartenvisualisierung wurde mit React Native Maps umgesetzt \cite{reactNativeMaps};
auf iOS kommt dabei das native \texttt{MapKit}-Framework zum Einsatz.
Routen werden als Great-Circle-Polylines gezeichnet und optional animiert.
\subsubsection{Anforderungen aus Pflichtenheft}
Gefordert sind Marker, Routen und Performancevorgaben für viele Flüge.
\subsubsection{Auswahl der Karten-Technologie}
React Native Maps bietet native Performance und einfache Integration in Expo.
\subsubsection{Implementierung der Flugrouten}
Die Route wird aus Airport-Koordinaten berechnet und als Polyline angezeigt.
\subsubsection{Live-Animation \& Performance-Optimierung}
Die Live-Position des Flugzeugs wird über Zeitstempel berechnet und in
regelmäßigen Intervallen aktualisiert.
\subsection{Umsetzung der Import-Funktionen}
Die Import-Module decken QR-Scan, OCR und E-Mail-Parsing ab.
\subsubsection{QR-Scan}
Boardingpässe werden per nativer iOS-Kamera gescannt, BCBP-Daten werden
geparst \cite{iataBCBP}.
\subsubsection{OCR-Dokumente/Bilder}
Texterkennung wird genutzt, wenn kein QR-Code vorhanden ist.
\subsubsection{E-Mail-Import}
Buchungsdaten werden aus E-Mails extrahiert und als Flight-Vorschläge angezeigt.
\subsection{Umsetzung der Benachrichtigungen}
Benachrichtigungen sind lokal geplant und mit Settings gekoppelt
\cite{expoNotifications}. Da Skyline primär für \textbf{iOS} entwickelt wurde,
bildet \textbf{APNs} (Apple Push Notification Service) die Zustellinfrastruktur;
das iOS-Betriebssystem steuert Anzeige, Zeitpunkt und Nutzerpermissions.
FCM (Android) ist als sekundäre, noch nicht vollständig getestete Erweiterung
vorgesehen.
\subsubsection{Reminder-Offsets}
Standard-Offsets wie T-24h und T-60m werden automatisch gesetzt.
\subsubsection{Quiet Hours}
Quiet Hours verschieben Notifications in erlaubte Zeitfenster; auf iOS
wird dabei die Focus-Mode-Infrastruktur des Betriebssystems berücksichtigt.
\subsubsection{Persistenz \& Reschedule}
Persistenz erlaubt Rescheduling bei App-Neustart; der iOS-Background-Scheduler
stellt sicher, dass geplante Notifications erhalten bleiben.
\subsection{Umsetzung der Datenverwaltung}
Supabase liefert Auth, Datenbank und Storage als zentrale Datenplattform
\cite{supabasePlatform,supabaseAuth,supabaseStorage}.
\subsubsection{Datenmodell \& Synchronisierung}
Flights sind die Kernentität; alle Module referenzieren diese Struktur.
\subsubsection{Dokumentenablage}
Dokumente werden in Storage-Buckets abgelegt und über Metadaten zugeordnet.
\subsubsection{Rechte \& Sicherheit}
RLS-Policies garantieren Zugriffskontrolle auf Daten- und Storage-Ebene
\cite{supabaseRLS}.
\subsection{Umsetzung der Gamification-Elemente}
Gamification dient der Motivation und Visualisierung von Fortschritt.
\subsubsection{Achievements}
Achievements werden bei Meilensteinen freigeschaltet.
\subsubsection{Fortschrittsdarstellung}
Progress-Elemente zeigen Nutzern ihre Reisehistorie und Statistiken.
\subsubsection{Feedback-Mechanismen}
Toast-Nachrichten und haptisches Feedback verbessern die Nutzerinteraktion.
\subsection{Teststrategie \& Validierung}
Tests prüfen Funktionalität, Stabilität und Genauigkeit der Kernmodule.
\begin{figure}[h]
  \centering
  \fbox{\parbox{0.8\textwidth}{TODO: Beispielhafte UI-Flows oder Testfall-Screenshot einfügen.}}
  \caption{TODO: Testfälle und UI-Flows}
  \label{fig:todo_tests}
\end{figure}
\subsubsection{Funktionstests (UI-Flows)}
Manuelle UI-Tests sichern die Hauptabläufe (Add Flight, Import, Trip Details);
alle primären Tests wurden auf iOS-Geräten durchgeführt.
\subsubsection{Reminder-Tests}
Reminder werden in Testfällen auf Offsets, Quiet Hours und Reschedule geprueft.
\subsubsection{Import-Tests}
Testfälle für QR, OCR und E-Mail-Import sichern robuste Datenaufnahme.
\subsubsection{Statistiken-Validierung}
Berechnungen für Distanz und Dauer werden mit Unit-Tests abgesichert.
