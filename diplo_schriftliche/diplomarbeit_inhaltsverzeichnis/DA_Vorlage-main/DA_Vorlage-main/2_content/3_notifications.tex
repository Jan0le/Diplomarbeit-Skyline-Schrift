%!TEX root = ../DA_MainDocument.tex
\chapter{Benachrichtigungs- und Erinnerungsmodul}\label{chapter:notifications}

\section{Grundlagen proaktiver Benachrichtigungssysteme}
Proaktive Benachrichtigungssysteme zeichnen sich dadurch aus, dass sie Nutzerinnen
und Nutzer nicht erst nach einer aktiven Abfrage informieren, sondern relevante
Ereignisse oder bevorstehende Aufgaben automatisch anzeigen. Im Kontext der
Reiseorganisation ist dies besonders wichtig, weil viele Handlungen zeitkritisch
sind und ein Versäumnis (z. B. fehlende Unterlagen oder verpasste Check-ins)
direkt zu Problemen führen kann. Durch proaktive Hinweise können solche
Fehlzustände reduziert und die organisatorische Belastung des Nutzers verringert
werden.

\subsection{Begriff „proaktiv“ und Abgrenzung zu reaktiven Systemen}
Proaktiv bedeutet, dass das System eigenständig handelt, ohne eine explizite
Anfrage des Nutzers. Es analysiert Kontextinformationen (z. B. Abflugzeit,
fehlende Dokumente) und löst rechtzeitig passende Hinweise aus. Reaktive
Systeme hingegen geben Informationen nur auf Nachfrage aus, wodurch der Nutzer
selbst ständig an notwendige Schritte denken und die Informationen aktiv
abrufen muss. In der Reiseorganisation ist der proaktive Ansatz vorteilhaft,
weil viele Aufgaben in engem zeitlichem Zusammenhang stehen und eine reine
„Abruf-Logik“ zu einer erhöhten Fehlerwahrscheinlichkeit führt.

\subsection{Erinnerungslasten in der Reiseorganisation}
Reiseabläufe bestehen aus mehreren zeitkritischen Einzelschritten, z. B.
Check-in, Boarding, Dokumentenbereitstellung oder das Nachreichen von Belegen.
Diese Schritte liegen oft zwischen anderen Verpflichtungen des Nutzers und
müssen zu festgelegten Zeitpunkten erfolgen. Ohne technische Unterstützung
entsteht eine hohe Erinnerungslast: Der Nutzer muss sich eigenständig erinnern,
Termine überwachen und relevante Informationen im Blick behalten. Proaktive
Erinnerungen reduzieren diese Belastung, indem sie Handlungen zur richtigen Zeit
auslösen und damit das Risiko von Fehlhandlungen oder Versäumnissen senken.

\subsection{Notification Fatigue (Überlastung)}
Ein zentrales Risiko proaktiver Benachrichtigungssysteme ist die sogenannte
„Notification Fatigue“: Werden zu viele Hinweise gesendet, steigt die
Wahrscheinlichkeit, dass Nutzer Benachrichtigungen ignorieren oder die Funktion
komplett deaktivieren. Studien zeigen, dass eine hohe Benachrichtigungsdichte die
Wahrnehmung von Relevanz reduziert und dadurch die Wirksamkeit der Hinweise
senkt \cite{pielot2014notifications}. Daraus folgt die Notwendigkeit klarer
Regeln: Hinweise müssen priorisiert, kontextabhängig gefiltert und so formuliert
werden, dass sie als hilfreich und nicht als störend wahrgenommen werden.

\subsection{Timing-Strategien in mobilen Apps}
Das Timing ist entscheidend für die Wirksamkeit von Benachrichtigungen. Hinweise
müssen früh genug erfolgen, damit der Nutzer handeln kann, aber spät genug,
damit sie im richtigen Kontext ankommen. Im Reisebereich haben sich praxistaugliche
Offsets etabliert, z. B. T-24h für den Check-in oder T-60m/T-30m für das Boarding.
Diese Zeitfenster geben ausreichend Reaktionszeit und orientieren sich an typischen
Reiseabläufen. In Skyline werden solche Offsets als projektspezifische Heuristiken
verwendet und mit weiteren Bedingungen kombiniert (z. B. „nur wenn Dokument fehlt“),
um die Relevanz jedes Hinweises sicherzustellen.

\section{Anforderungen im Reise Kontext}
Im Reise-Kontext sind Benachrichtigungen besonders zeitkritisch und müssen
an die zuvor angegebenen Ankunfts- und Abflugszeiten der Kunden gekoppelt
werden. Ohne diese Kopplung entstehen unzuverlässige Erinnerungen, weil
Hinweise entweder zu spät oder ohne relevanten Bezug zum tatsächlichen
Reiseverlauf erscheinen. In Skyline basiert die Planung der Reminder daher
auf manuellen Eingaben sowie auf Daten aus Flugtickets, QR-Codes und PDFs.
Eine automatische Aktualisierung der Zeiten ist nicht umgesetzt, da dafür
kostenpflichtige Echtzeit-APIs erforderlich wären.

\subsection{Checkin Reminder (T24h)}
Check-in Reminders informieren typischerweise 24 Stunden vor Abflug und geben
dem Nutzer ausreichend Zeit, den Online-Check-in durchzuführen, Sitzplätze
zu wählen und eventuell notwendige Dokumente zu prüfen. Dadurch wird das
Risiko reduziert, dass der Check-in verpasst wird oder am Reisetag Stress
entsteht.
\begin{figure}[h]
  \centering
  \fbox{\parbox{0.8\textwidth}{Platzhalter: Screenshot der Check-in Benachrichtigung (T-24h).}}
  \caption{Beispiel: Check-in Reminder}
  \label{fig:reminder_checkin}
\end{figure}

\subsection{Boarding Reminder (T60m / T30m)}
Boarding Hinweise werden kurz vor dem Boarding ausgeliefert, typischerweise
60 bzw. 30 Minuten vor Abflug. Diese Erinnerungen sind besonders relevant bei
engen Umsteigezeiten oder bei größeren Airports, da sie den Nutzer rechtzeitig
an Gate-Wechsel, Boarding-Zeiten und das Finden des richtigen Abflugbereichs
erinnern.
\begin{figure}[h]
  \centering
  \fbox{\parbox{0.8\textwidth}{Platzhalter: Screenshot der Boarding Benachrichtigung (T-60m / T-30m).}}
  \caption{Beispiel: Boarding Reminder}
  \label{fig:reminder_boarding}
\end{figure}

\subsection{Missing Docs Reminder (T12h)}
Fehlende Unterlagen sollen möglichst früh erkannt werden, damit Nutzer noch
handeln können. Der Reminder wird deshalb ca. 12 Stunden vor Abflug gesendet,
sofern wichtige Dokumente (z. B. Boardingpass, Buchungsbestätigung oder
Rechnung) fehlen. Dadurch bleibt genug Zeit, die Unterlagen nachzureichen.
\begin{figure}[h]
  \centering
  \fbox{\parbox{0.8\textwidth}{Platzhalter: Screenshot der Dokumente-Fehlen Benachrichtigung (T-12h).}}
  \caption{Beispiel: Missing Docs Reminder}
  \label{fig:reminder_missing_docs}
\end{figure}

\subsection{Receipt Reminder (T+2h)}
Nach der Ankunft wird ein Reminder für Belege gesetzt, um die Abrechnung
zu unterstützen. Der zeitliche Abstand von etwa zwei Stunden ist bewusst
gewählt, damit der Nutzer zuerst den Reiseabschluss erledigen kann und danach
ruhig die Belege hochlädt oder fotografiert.
\begin{figure}[h]
  \centering
  \fbox{\parbox{0.8\textwidth}{Platzhalter: Screenshot der Beleg-Erinnerung (T+2h).}}
  \caption{Beispiel: Receipt Reminder}
  \label{fig:reminder_receipt}
\end{figure}

\subsection{Kontextabhängige Hinweise (z. B. fehlende Dokumente)}
Hinweise werden nur gesendet, wenn die Kontextbedingungen stimmen. Ein Beispiel
ist der Fall, dass wichtige Dokumente fehlen und der Abflug innerhalb der
nächsten 48 Stunden liegt. Dadurch werden Benachrichtigungen auf relevante
Situationen beschränkt und die Nutzer nicht mit irrelevanten Meldungen
überlastet.
\begin{figure}[h]
  \centering
  \fbox{\parbox{0.8\textwidth}{Platzhalter: Screenshot eines kontextabhängigen Hinweises.}}
  \caption{Beispiel: Kontextabhängige Benachrichtigung}
  \label{fig:reminder_context}
\end{figure}

\subsection{Triggerberechnung aus departureAt/arrivalAt}
Triggerzeiten werden aus Abflug- und Ankunftszeiten abgeleitet und als Offsets
berechnet.
\subsection{Speicherung \& Persistenz (local + server)}
Lokale IDs werden gespeichert; zusaetzlich werden geplante Reminders serverseitig
persistiert, um Rescheduling zu ermoeglichen.
\subsection{Rescheduling nach AppStart}
Beim App-Start werden ausstehende Reminders geladen und neu geplant.
\subsection{DeepLinks zu Trip Details}
Benachrichtigungen enthalten Deeplinks, die direkt in die Flugdetails fuehren.
\subsection{Fehlerhandling (fehlende Zeiten, invalid data)}
Fehlende oder ungueltige Zeiten verhindern Scheduling und werden abgefangen.
\subsection{DebugAnsicht (Pending Notifications)}
Eine Debug-Ansicht zeigt geplante und serverseitige Notifications fuer Tests.
\section{Implementierung in Skyline}
Die Implementierung nutzt Expo Notifications und eine eigene Registry fuer
Persistenz \cite{expoNotifications}.
\subsection{ReminderOffsets \& SchedulingFlow}
Beim Speichern eines Flugs werden die Offsets geprueft und geplant.
\subsection{Integration beim FlightSave}
Die Scheduling-Logik wird automatisch beim Flug-Speichern angestossen.
\subsection{Cancel/Reschedule bei Updates}
Bei Aenderungen werden alte Reminders gecancelt und neu gesetzt.
\subsection{PushIntegration (EAS / Expo Tokens)}
Push-Benachrichtigungen sind konzeptionell vorbereitet; fuer Produktion
braucht es FCM/APNs via EAS.
\section{Bewertung der Wirkung}
Die Wirkung wird ueber Zuverlaessigkeit, Effizienz und Nutzerakzeptanz bewertet.
\subsection{Zuverlaessigkeit als KPI}
KPI: Anteil der Fluege ohne kritische Fehlzustaende (z. B. fehlende Unterlagen).
\subsection{Effizienz als KPI}
KPI: Reduktion der Suchzeit nach Dokumenten und Anzahl manueller Schritte.
\subsection{Nutzerakzeptanz}
Akzeptanz wird ueber qualitative Rueckmeldungen und Settings-Nutzung beurteilt.
\section{Ergebnis}
Das Modul erhoeht die Verlaesslichkeit der Reiseorganisation, wenn Timing und
Relevanz stimmen.
\subsection{Reduktion kritischer Fehlzustaende}
Hinweise auf Check-in und fehlende Dokumente reduzieren Fehler.
\subsection{Effizienzsteigerung}
Weniger Suchaufwand und klarere Ablaufe steigern die Effizienz.
\subsection{Gesamtbewertung}
Die Kombination aus Reminder-Offsets, Quiet Hours und Deeplinks liefert einen
messbaren Mehrwert.
